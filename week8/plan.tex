\documentclass{whrartcl}

\title{Set theory: Week 8}
\author{Dominik Wehr, University of Gothenburg}
\date{Updated: \today}

\usepackage[inference]{semantic}
\usepackage{mathpartir}
\usepackage{xcolor}
\usepackage{url}
\usepackage{bussproofs}

% Fix \mu in the bibliography
\DeclareUnicodeCharacter{03BC}{$\mu$}
\usepackage[
  style=numeric-comp,
  sorting=nyt,
  maxbibnames=99,
  backref=false,
  doi=false,
  isbn=false,
  url=false,
  eprint=false,
  backend=biber
  ]{biblatex}
\bibliography{bibliography}

\newcommand{\NN}{\mathbb{N}}
\newcommand{\ZZ}{\mathbb{Z}}
\newcommand{\QQ}{\mathbb{Q}}
\newcommand{\RR}{\mathbb{R}}
\newcommand{\BB}{\mathbb{B}}
\newcommand{\FF}{\mathcal{F}}
\newcommand{\PP}{\textbf{P}}
\newcommand{\DD}{\mathcal{D}}
\newcommand{\II}{\mathcal{I}}
\newcommand{\TT}{\mathcal{T}}
\newcommand{\Aa}{\mathcal{A}}
\newcommand{\Bb}{\mathcal{B}}
\newcommand{\Vv}{\mathcal{V}}
\newcommand{\WW}{\textbf{W}}
\newcommand{\LL}{\mathcal{L}}
\newcommand{\CC}{\mathcal{C}}
\newcommand{\pow}{\mathcal{P}}
\newcommand{\form}{\textsc{Form}}
\newcommand{\dom}{\text{dom}}
\newcommand{\ran}{\text{ran}}
\newcommand{\IS}{\text{IS}}
\newcommand{\ord}{\text{Ord}}
\newcommand{\Exp}{\text{Exp}}
\newcommand{\supp}{\text{supp}}

\newcommand{\ol}[1]{\overline{#1}}
\newcommand{\abs}[1]{|#1|}
\newcommand{\uh}{\upharpoonright}
\newcommand{\irule}[3]{\inference[\textsc{#1}]{#2}{#3}}
\newcommand{\cls}[1]{\llbracket #1 \rrbracket}

\newcommand{\TODO}[1]{{\color{red}[TODO: #1]}}
\newcommand{\needcite}{{\color{red}[?]}}

\begin{document}

\maketitle

\section{Lecture 1}

\subsection{Hartog's theorem}

\begin{theorem}[Hartog's theorem]
  For any set $X$, there exists an ordinal $\alpha$ such that $\alpha
  \not\preceq X$.
\end{theorem}
\begin{proof}
  We shall construct the least such ordinal, which will be
  \[
    \alpha := \{\beta \mid \beta \text{ is an ordinal and } \beta \preceq X\}.
  \]
  First, suppose $\alpha$ was a set and an ordinal, then clearly if $\alpha
  \preceq X$ then $\alpha \in \alpha$, contradicting Foundation.

  We begin by proving that $\alpha$ is a set. The following set exists by
  separation:
  \[
    W := \{R \subseteq X \times X \mid (\supp(R), R) \text{ is a well-order}\}
  \]
  The following set exists by replacement (we have shown $\ord$ to be a
  class-level function on $W$ last week):
  \[
    A := \{\ord(\supp(R), R) \mid R \in W\}.
  \]
  We now claim that $\beta \in A$ iff $\beta$ is an ordinal and $\beta \preceq
  X$, i.e. that $A = \alpha$:
  \begin{description}
  \item[$\to$:] If $\beta \in A$ then $\beta = \ord(\supp(R), R)$ for $R \in W$.
    This means $\beta$ is an ordinal such that $(\beta, \in) \approx (\supp(R),
    R)$. This means that $\beta \approx \supp(R) \subseteq X$ as desired.
  \item[$\leftarrow$:] Let $\beta$ be an ordinal such that there is an injection
    $i : \beta \to X$. Consider the following relation
    \[
      R := \{(i(\gamma), i(\delta)) \mid \gamma, \delta \in \beta\}
    \]
    then $i$ is an order-isomorphism between $(\beta, \in)$
    and $(\ran(i), R)$ per construction of $R$. This means that $(\ran(i), R)$ is
    a well-order and thus that $R \in W$. Then $\beta = \ord(\ran(i),
    R)$ and we have $\beta \in A$.
  \end{description}

  Lastly we have to prove that $\alpha$ is an ordinal. It is a set of ordinals,
  so it remains to show transitivity: Let $\gamma \in \beta \in \alpha$ then
  $\gamma \subseteq \beta \preceq X$ and thus $\gamma \preceq X$.
\end{proof}

\begin{remark}
  Recall: 
  \emph{Zorn's lemma} ZL states: Let $X$ be partially ordered by $\sqsubseteq\;
  \subseteq X \times X$ in such a way that every $\sqsubseteq$-chain $Y \subseteq X$
  has an upper bound in $X$. Then $X$ has a $\sqsubseteq$-maximal element.
\end{remark}

\begin{theorem}
  The axiom of choice implies Zorn's lemma.
\end{theorem}
\begin{proof}
  Suppose $(X, \sqsubseteq)$ was a partial order and chain-closed. Suppose $X$
  had no $\sqsubseteq$-maximal element. Let $Y \subseteq X$ be a chain, then
  there exists an upper bound $Y \sqsubseteq x$. Because there are no maximal
  elements, there must be $x \sqsubset y$ and thus every chain has a strict
  upper bound $Y \sqsubset y$.
  By the AC, there exists a choice
  function $c : \Pi Y \in \{Y \subseteq X \mid Y \text{ a chain} \}.~\{y \in X
  \mid Y \sqsubset y\}$.

  By Hartog's
  theorem, there exists an ordinal $\alpha$ such that $\alpha \not\preceq X$.
  We shall now construct an injection $\alpha \to X$, contradicting $X
  \not\approx \alpha$. Consider the recursive function $i : \alpha \to X$:
  \[
    i(\beta) :=
    \begin{cases}
      c(\ran(i \uh \beta)) & \text{ if } i \uh \beta \text{ an $\sqsubset$-chain} \\
      \emptyset & \text{otherwise}
    \end{cases}
  \]

  First, observe that for every $\beta \in \alpha$, we have $i(\beta) \in X$ and
  for any $\ran(i \uh \beta) \sqsubset i(\beta)$. Proof per
  induction on $\beta$: First, observe that $\ran(i \uh \beta)$ is a chain
  as $i(\delta) \sqsubset i(\gamma)$ if $\delta \in \gamma \in \beta$, per IH of
  $\gamma$.
  Thus $i(\beta) = c(\ran(i \uh \beta)) \in X$ and thus $\ran(i \uh \beta) \sqsubset
  i(\beta)$ as claimed.

  From $\ran(i \uh \beta) \sqsubset i(\beta)$ it also follows directly that $i :
  \alpha \to X$ is an injection.
\end{proof}

\subsection{The $\aleph$s}

\begin{definition}
  We call $\alpha$ \emph{initial} if for all $\beta \in \alpha$ we have $\beta \prec \alpha$.
\end{definition}

\begin{lemma}
  For every ordinal $\alpha$ there is a \emph{unique} initial ordinal
  $\beta \approx \alpha$.
\end{lemma}
\begin{proof}
  Consider
  \(
    B := \{\gamma \leq \alpha \mid \gamma \approx \alpha\}.
  \)
  We claim that $\beta := \min B$ is as desired. Clearly $\beta \approx \alpha$.
  \begin{description}
  \item[initial:] $\beta$ is infinite at $\alpha$ is. Suppose there was $\gamma
    \in \beta$ such that $\beta \preceq \gamma$, then $\alpha \approx \beta
    \preceq \gamma$ and thus $\gamma \approx \alpha$. But then $\gamma \in B$ and
    thus $\beta \neq \min B$.
  \item[unique:] Suppose there was another initial $\gamma \leq \alpha$ with
    $\gamma \approx \alpha$. By linearity, it suffices to w.l.o.g.\ rule out
    $\gamma \in \beta$. If this was the case, $\gamma \prec \beta$ but this
    cannot be the case as $\gamma \approx \alpha \approx \beta$.
  \end{description}
\end{proof}

\begin{definition}
  We define the \emph{sequence of infinite initial ordinals} $\aleph_\bullet$ by
  transfinite recursion:
  \begin{align*}
    \aleph_0 & := \omega \\
    \aleph_{\alpha^+} & := \text{the least initial } \gamma \text{ such that } \aleph_\alpha \prec \gamma \\
    \aleph_{\lambda} & := \bigcup \{\aleph_\gamma \mid \gamma \in \lambda\}
  \end{align*}
  where $\lambda$ is a limit.
\end{definition}

\begin{lemma}
  For any ordinal $\alpha$, $\aleph_\alpha$ is initial.
\end{lemma}
\begin{proof}
  Induction on $\alpha$. The $0$ and successor cases are obvious. Thus, let
  $\alpha = \lambda$ be a limit such that all $\aleph_\beta$ for $\beta \in
  \lambda$ are initial. We have to show that is $\gamma \in \lambda$ then
  $\lambda \not\preceq \gamma$. We know that $\gamma \in \aleph_\beta$ for $\beta
  \in \gamma$. If $\lambda \preceq \gamma$ then $\aleph_\beta \preceq \lambda
  \preceq \gamma$, contradicting the initiality of $\aleph_\beta$.
\end{proof}

\begin{fact}
  If $\alpha \in \beta$ then $\aleph_\alpha \prec \aleph_\beta$.
\end{fact}

\begin{lemma}
  Let $\alpha$ be infinite and initial. Then there exists $\beta$ such that $\alpha = \aleph_\beta$.
\end{lemma}
\begin{proof}
  We first prove that for any $\alpha$, $\alpha \leq \aleph_\alpha$.
  \begin{description}
  \item[$0$:] Clearly $0 \leq \omega$.
  \item[$\beta^+$:] If $\beta^+$ is finite then $\beta^+ \in \aleph_0 \leq
    \alpha_{\beta^+}$. If $\beta^+$ is infinite then $\beta^+ \approx \beta \leq
    \aleph_\beta \prec \aleph_{\beta^+}$.
  \item[$\lambda$ limit:] Observe that for all $\beta \in \lambda$ we have
    $\beta \leq \aleph_\beta \leq \aleph_\lambda$ and thus $\lambda \subseteq
    \aleph_\lambda$, meaning $\lambda \leq \aleph_\lambda$.
  \end{description}

  Consider
  \[
    X := \{\beta \leq \alpha \mid \aleph_\beta \leq \alpha\}.
  \]
  Clearly, $X$ is transitive as $\aleph_\gamma \leq \aleph_\beta$ for $\gamma
  \in \beta$, and thus $X$ is some ordinal. We claim that it is a successor:
  \begin{description}
  \item[$0 \in X$:] As $\alpha$ is infinite, $\omega = \aleph_0 \leq \alpha$.
  \item[$X$ no limit:] Suppose $X$ was a limit. Because $\aleph_\beta \leq
    \alpha$ for all $\beta \in X$, this means that $\aleph_X = \bigcup
    \{\aleph_\beta \mid \beta \in X\} \leq \alpha$.
    Because $X \leq \alpha$ per construction, this
    would mean $X \in X$, a contradiction.
  \end{description}

  Hence $X = \beta^+$ for some ordinal $\beta$. We claim $\alpha =
  \aleph_\beta$. We know that $\aleph_\beta \leq \alpha < \aleph_{\beta^+}$
  because $\beta^+ \not\in X$. Because $\aleph_\beta, \alpha, \aleph_{\beta^+}$
  are all initial and $\aleph_{\beta^+}$ is the least initial ordinal larger
  than $\aleph_\beta$, we obtain $\alpha = \aleph_\beta$.
\end{proof}

\newpage

\section{Lecture 2}

\subsection{Defining $\abs{X}$}

\begin{definition}
  Let $X$ be a set. Its cardinality $\abs{X}$ is any initial ordinal $\alpha$
  with $X \approx \alpha$.
\end{definition}

\begin{remark}
  If $\abs{X}$ exists then $\abs{X} = \aleph_\alpha$ for some ordinal $\alpha$.
\end{remark}

\begin{lemma}
  Let $\abs{X}, \abs{Y}$ exist. Then $\abs{X} = \abs{Y}$ iff $X \approx Y$.
\end{lemma}
\begin{proof}
  \
  \begin{description}
  \item[$\to$:] Then $X \approx \abs{X} = \abs{Y} \approx Y$.
  \item[$\leftarrow$:] Then $\abs{X} \approx X \approx Y \approx \abs{Y}$ and
    because $\abs{X}, \abs{Y}$ are initial, this means $\abs{X} = \abs{Y}$.
  \end{description}
\end{proof}

\begin{corollary}
  For any set $X$, if $\abs{X}$ exists it is unique.
\end{corollary}

\begin{theorem}
  The following statement in equivalent to the AC: For every set $X$, $\abs{X}$ exists.
\end{theorem}
\begin{proof}
  \
  \begin{description}
  \item[AC $\to$ Statement:] The WPO yields a well-order $(X, <)$. Then
    there exists $\alpha := \ord(X, <) \approx (X, <)$. Ultimately, there
    exists an initial ordinal $\beta \approx \alpha \approx X$.
  \item[Statement $\to$ AC:] We prove the WPO. Let $X$ be a set, then there is
    an ordinal $X \approx \abs{X}$. Let $b : X \to \abs{X}$ be the bijection,
    then $X$ is well-ordered by
    \[
      x < x' \text{ iff } b(x) \in b(x').
    \]
  \end{description}
\end{proof}

\begin{remark}
  We shall now study the arithmetic of cardinal numbers. To do this, we will
  pretend that all cardinalities we treat exist. Formally, this means we either
  assume AC, which just grants the existence of all cardinalities, or we make
  all statements and definitions conditional on the existence of all
  cardinalities involved.
\end{remark}

\subsection{Cardinal arithmetic}

\begin{definition}
   Let $X$ and $Y$ be sets. We define
   \begin{align*}
     \abs{X} + \abs{Y} & := \abs{X + Y} = \abs{X \times \{0\} \cup Y \times \{1\}} \\
     \abs{X} \cdot \abs{Y} & := \abs{X \times Y} \\
     \abs{X}^{\abs{Y}} & := \abs{X^Y}
   \end{align*}
\end{definition}

\begin{remark}
  We extend this notation to the $\aleph$s. For example, we often write
  $\aleph_\alpha^{\aleph_\beta}$ to formally mean
  $\abs{\aleph_\alpha}^{\abs{\aleph_\beta}}$ because we know that
  $\abs{\aleph_\gamma} = \aleph_\gamma$.
\end{remark}

\begin{lemma} Although $\aleph_0 = \NN = \omega$,
  \[
    \text{(as cardinals)  } \aleph_0^{\aleph_0} \not\approx \omega^\omega
    \text{  (as ordinals)}.
  \]
\end{lemma}
\begin{proof}
  Observe that $\omega^\omega \approx \Exp(\omega, \omega) = \{f : \omega \to
  \omega \mid \{n \in \omega \mid f(n) > 0\} \text{ is finite}\} \approx \omega$.

  On the other hand, $\aleph_0^{\aleph_0} \approx 2^{\aleph_0} \succ \aleph_0 = \omega$.
\end{proof}

\begin{theorem}
  If $X, Y$ are infinite then
  \[
    \abs{X} \cdot \abs{Y} = \max \{\abs{X}, \abs{Y}\} = \abs{X} + \abs{Y}.
  \]
\end{theorem}
\begin{proof}
  We first prove that $\alpha \times \alpha \approx \alpha$ for all infinite
  ordinals. Induction on $\alpha$.
  \begin{description}
  \item[$\omega$:] We have shown $\NN \approx \NN \times \NN$ before.
  \item[$\beta^+$:] Then $\beta^+ \approx \beta \stackrel{IH}{\approx} \beta \times \beta
    \approx \beta^+ \times \beta^+$ because $\beta \approx \beta^+$.
  \item[$\lambda$ limit:] If $\lambda$ is not initial then there is $\gamma \in
    \lambda$ with $\lambda \approx \gamma \approx \gamma \times \gamma \approx
    \lambda \times \lambda$.

    Thus, suppose $\lambda$ was initial. We begin by defining a well-ordering $(\lambda
    \times \lambda, \sqsubset)$:
    \begin{align*}
      (\alpha, \beta) \sqsubset (\alpha', \beta') & \text{ if } \max \{\alpha, \beta\} < \max \{\alpha', \beta'\} \\      & \text{ or } \max \{\alpha, \beta\} = \max \{\alpha', \beta'\} \text{ and } \alpha < \alpha' \\
      & \text{ or } \max \{\alpha, \beta\} = \max \{\alpha', \beta'\} \text{ and } \alpha = \alpha' \text{ and } \beta < \beta'
    \end{align*}
    This is a well-order and furthermore, we have that for any $\gamma \in \lambda$,
    $(\gamma \cdot \gamma, \in) \approx \IS_{\sqsubset}((\gamma \cdot \gamma, 0))$
    [Exercise 7.69]. Thus $(\bigcup \{\gamma \cdot \gamma \mid \gamma \in
    \lambda\}, \in) \approx (\lambda \times \lambda, \sqsubset)$.

    We know that $(\gamma \cdot \gamma, \in) \approx (\gamma \times \gamma,
    <_\otimes)$. We thus know that $\gamma \stackrel{IH}{\approx} \gamma \times \gamma \approx
    \gamma \cdot \gamma$ as sets. Because $\gamma \prec \lambda$, it follows
    that $\gamma \cdot \gamma \prec \lambda$. Hence
    \[
      \lambda \times \lambda \approx \bigcup \{\gamma \cdot \gamma \mid \gamma
      \in \lambda\} \leq \lambda.
    \]
    Trivially, $\lambda \prec \lambda \times \lambda$ thus the claim follows by CSB.
  \end{description}

  Now suppose, w.l.o.g., that $\abs{X} = \aleph_\alpha \preceq \aleph_\beta = \abs{Y}$. Then
  \[
    \aleph_\beta \preceq \aleph_\beta + \aleph_\alpha \preceq \aleph_\beta +
    \aleph_\beta \approx \aleph_\beta \times 2 \preceq \aleph_\beta \times \aleph_\beta \approx \aleph_\beta
  \]
  and
  \[
    \aleph_\beta \preceq \aleph_\beta \cdot \aleph_\alpha \preceq \aleph_\beta \cdot \aleph_\beta \approx \aleph_\beta \times \aleph_\beta \approx \aleph_\beta.
  \]
\end{proof}

\subsection{Continuum Hypothesis, revisited}

\begin{definition}
  We defined the continuum hypothesis as follows: For any $X \subseteq \RR$ one
  of the following applies:
  \begin{itemize}
  \item $X$ is finite
  \item $X \approx \NN$
  \item $X \approx \RR$
  \end{itemize}
\end{definition}

\begin{lemma}
  The continuum hypothesis can be stated as $2^{\aleph_0} = \aleph_1$.
\end{lemma}
\begin{proof}
  Let $\RR = 2^\NN \approx \aleph_1$ and let $b : \RR \to \aleph_1$ witness
  this. Then any $X \subseteq \RR$ can be well-ordered by
  \[
    x < x' \text{ iff } b(x) \in b(x').
  \]
  Consider $\alpha := \ord(X, <)$. There exists an initial $\beta \approx
  \alpha$. Clearly, $\beta \leq \aleph_1$ meaning one of the following holds:
  \begin{itemize}
  \item $\beta$ is finite or
  \item $\beta = \aleph_0$ or
  \item $\beta = \aleph_1$.
  \end{itemize}
\end{proof}

\begin{definition}
  The \emph{generalized continuum hypothesis} states that $2^{\aleph_\alpha} =
  \aleph_{\alpha + 1}$.
\end{definition}

\begin{theorem}
  We know that $2^{\aleph_0} \not\approx \aleph_\omega$.
\end{theorem}
\begin{proof}
  Suppose $2^{\aleph_0} = 2^\NN \approx \aleph_\omega$. Because $(2^\NN) \approx (2^\NN)^\NN$,
  we can obtain a surjection $b : \aleph_\omega \to (\NN \to \aleph_\omega)$. Then we define
  a function $g : \NN \to \aleph_\omega$ such that $g \not\in \ran(b)$ as
  follows
  \[
    g(n) := \min (\aleph_\omega \setminus \{f(\beta)(n) \mid \beta \in \aleph_n\}).
  \]
  Because $\aleph_n \prec \aleph_\omega$, the set whose minimum $g(n)$ is
  defined as is never empty and thus $g$ well-defined. We show that $g \not\in
  \ran(b)$: Suppose $g = f(\alpha)$ with $\alpha \in \aleph_\omega$, meaning
  $\aleph \in \aleph_n$. But then $f(\alpha)(n) \neq g(n)$.
\end{proof}

\printbibliography{}

\end{document}
