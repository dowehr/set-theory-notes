\documentclass{whrartcl}

\title{Set theory: Week 4}
\author{Dominik Wehr, University of Gothenburg}
\date{Updated: \today}

\usepackage[inference]{semantic}
\usepackage{mathpartir}
\usepackage{xcolor}
\usepackage{url}
\usepackage{bussproofs}

% Fix \mu in the bibliography
\DeclareUnicodeCharacter{03BC}{$\mu$}
\usepackage[
  style=numeric-comp,
  sorting=nyt,
  maxbibnames=99,
  backref=false,
  doi=false,
  isbn=false,
  url=false,
  eprint=false,
  backend=biber
  ]{biblatex}
\bibliography{bibliography}

\newcommand{\NN}{\mathbb{N}}
\newcommand{\ZZ}{\mathbb{Z}}
\newcommand{\QQ}{\mathbb{Q}}
\newcommand{\RR}{\mathbb{R}}
\newcommand{\BB}{\mathbb{B}}
\newcommand{\PP}{\textbf{P}}
\newcommand{\DD}{\mathcal{D}}
\newcommand{\II}{\mathcal{I}}
\newcommand{\TT}{\mathcal{T}}
\newcommand{\Aa}{\mathcal{A}}
\newcommand{\Bb}{\mathcal{B}}
\newcommand{\Vv}{\mathcal{V}}
\newcommand{\WW}{\textbf{W}}
\newcommand{\LL}{\mathcal{L}}
\newcommand{\CC}{\mathcal{C}}
\newcommand{\pow}{\mathcal{P}}
\newcommand{\form}{\textsc{Form}}
\newcommand{\dom}{\text{dom}}
\newcommand{\ran}{\text{ran}}

\newcommand{\ol}[1]{\overline{#1}}
\newcommand{\abs}[1]{|#1|}
\newcommand{\uh}{\upharpoonright}
\newcommand{\irule}[3]{\inference[\textsc{#1}]{#2}{#3}}
\newcommand{\cls}[1]{\llbracket #1 \rrbracket}

\newcommand{\TODO}[1]{{\color{red}[TODO: #1]}}
\newcommand{\needcite}{{\color{red}[?]}}

\begin{document}

\maketitle

\section{Lecture 1}

\emph{Material:} Section 2.4

\subsection{Integers}

\emph{Idea:} Represent $z \in \ZZ$ by $a - b = z$ with $a, b \in \NN$.

\emph{Problem:} These $a, b \in \NN$ are not unique for $z \in \ZZ$.

\begin{definition}[Equivalence class]
  \
  A relation $\sim \;\subseteq A \times A$ is an \emph{equivalence relation} if it is
  \begin{itemize}
  \item \emph{reflexive:} $a \sim a$,
  \item \emph{symmetric:} $a \sim b$ entails $b \sim a$, and
  \item \emph{transitive:} $a \sim b$ and $b \sim c$ entail $a \sim c$.
  \end{itemize}
  Given an equivalence relation $\sim \; \subseteq A \times A$ an \emph{equivalence
  class} is:
  \[
    \llbracket a \rrbracket := \{b \in A \mid a \sim b\}
  \]
  and the \emph{quotient of $A$ by $\sim$} is $A_{/ \sim} := \{\llbracket a
  \rrbracket \mid a \in A\}$.
\end{definition}

\begin{lemma}
  $a \sim b$ iff $\cls{a} = \cls{b}$.
\end{lemma}
\begin{proof}
  \
  \begin{description}
  \item[$\to$:] We show $\cls{a} \subseteq \cls{b}$. Let $c \in \cls{a}$ then $c
    \sim a \sim b$ and thus $c \in \cls{b}$.
  \item[$\leftarrow$:] We know that $a \in \cls{a} = \cls{b}$ and thus $a \sim
    b$ per definition of $\cls{b}$.
  \end{description}
\end{proof}

\begin{definition}[Integers]
  We fix $\ZZ := (\NN \times \NN)_{/ \sim}$ for
  \(
    (a, b) \sim (c, d) \text{ iff } a + d = c + b.
  \)
\end{definition}

\begin{lemma}
  This is an equivalence relation.
\end{lemma}
\begin{proof}
  \
  \begin{description}
  \item[Reflexivity:] $(a, b) \sim (a, b) \leftrightarrow a + b = a + c$
  \item[Symmetry:] $(a, b) \sim (c, d) \leftrightarrow a + d = c + b
    \leftrightarrow c + b = a + d \leftrightarrow (c, d) \sim (a, b)$
  \item[Transitivity:] Let $(a, b) \sim (c, d) \sim (e, f)$, then $a + d = c +
    b$ and $c + f = e + d$. Together this yields $a + d + c + f = c + b + e +
    d$, leaving $a + f = e + b$ and thus $(a, b) \sim (e, f)$.
  \end{description}
\end{proof}

\begin{example}
  For $n \in \NN$ we have $-n := \{(k, k + n) \mid k \in \NN\}$.
\end{example}

\begin{definition}
  Let $\sim \subseteq X \times X$ be an equivalence relation. A function $f : X
  \times \ldots \times X
  \to Y$ is \emph{congruent} on $\sim$ if $a_1 \sim b_1, \ldots, a_n \sim b_n$
  entails $f(a_1, \ldots, a_n) = f(b_1, \ldots, b_n)$. A relation $R \subseteq X
  \times \ldots \times X$ is congruent on $\sim$ if for $a_1 \sim b_1, \ldots,
  a_n \sim b_n$ we have $(a_1, \ldots, a_n) \in R$ iff $(b_1, \ldots, b_n) \in R$.
\end{definition}

\begin{lemma}
  Let $f : X \times \ldots \times X \to Y$ be congruent on $\sim$. Then
  \[
    g \coloneq \{(A_1, \ldots, A_n, y) \in X_{/ \sim} \times \ldots \times X_{/
      \sim} \times Y \mid \exists a_1 \in A_1, \ldots, a_n \in A_n.~y = f(a_1,
    \ldots, a_n) \}
  \]
  is a function. We may thus write $g(\cls{a_1}, \ldots, \cls{a_n}) := f(a_1,
  \ldots, a_n)$.
\end{lemma}
\begin{proof}
  We must show that $g$ is a function. Thus fix $A_1, \ldots, A_n \in X_{/
    \sim}$ and $y, y' \in Y$ such that $(A_1, \ldots, A_n, y), (A_1, \ldots,
  A_n, y') \in g$. That means there are $a_i, b_i \in A_i$ such that $y = f(a_1,
  \ldots, a_n)$ and $y' = f(b_1, \ldots, b_n)$. Because $a_i, b_i \in A_i \in
  X_{/ \sim}$, we know that $a_i \sim b_i$. As f is congruent on $\sim$ that
  means $y = f(\vec{a}) = f(\vec{b}) = y'$.
\end{proof}

\begin{remark}
  Similarly, congruent relations induce well-defined relations on the quotient.
\end{remark}

\begin{definition}[Operations]
  We define
  \begin{itemize}
  \item $\cls{(a, b)} <_\ZZ \cls{(c, d)}$ if $a + d < b + c$
  \item $\cls{(a, b)} +_\ZZ \cls{(c, d)} := \cls{(a + c, b + d)}$
  \item $\cls{(a, b)} \cdot_\ZZ \cls{(c, d)} := \cls{(ac + bd,bc + ad)}$
  \item $\tilde{n} := \cls{(n, 0)}$ for $n \in \NN$
  \end{itemize}
\end{definition}

\begin{proposition}
  Let $(a, b) \sim (a', b')$ and $(c, d) \sim (c', d')$ then $a + d < b + c$ iff
  $a' + d' < b' + c'$.
\end{proposition}
\begin{proof}
  We know that $(1)~a + b' = a' + b$ and $(2)~c + d' = c' + d$. Then
  \begin{align*}
    a + d &< b + c & +(1) \\
    a + a' + b + d &< a + b + b' + c \\
    a' + d &< b' + c & +(2) \\
    a' + c + d + d' &< b' + c + c' + d \\
    a' + d' &< b' + c'
  \end{align*}
\end{proof}

\subsection{Rationals}

\emph{Idea:} $q \in \QQ$ is $(a, b) \in \ZZ \times \ZZ^+$ such that $q = \frac a
b$ where $\ZZ^+ = \{z \in \ZZ \mid z > \tilde{0}\}$.

\begin{definition}
  We take $\QQ := (\ZZ \times \ZZ^+)_{/ \sim}$ where $(a, b) \sim (c, d)$ if $ad
  = bc$.

  Further, we define the following:
  \begin{itemize}
  \item $\cls{(a, b)} <_\QQ \cls{(c, d)}$ if $ad <_\ZZ bc$
  \item $\cls{(a, b)} +_\QQ \cls{(c, d)} := \cls{(ad + bc, bd)}$
  \item $\cls{(a, b)} \cdot_\QQ \cls{(c, d)} := \cls{(ac, bd)}$
  \item $\hat{z} := \cls{(z, \tilde{1})}$ for $z \in \ZZ$
  \end{itemize}
\end{definition}

\begin{theorem}[Density]
  For $\cls{(a, b)} < \cls{(c, d)} \in \QQ$ there is $\cls{(x,y)} \in \QQ$ such
  that $\cls{(a, b)} < \cls{(x, y)} < \cls{(c, d)}$.
\end{theorem}
\begin{proof}
  \emph{Intuitively}: Pick $\frac 1 2 (\frac a b + \frac c d)$.

  Formally, that is
  \[
    \cls{(1, 2)} \cdot (\cls{(a, b)} + \cls{(c, d)}) = \cls{(1, 2)} \cdot
    \cls{(ad + bc, bd)} = \cls{(ad + bc, 2bd)}
  \]

  Now
  \[
    \cls{(a, b)} < \cls{(ad + bc, 2bd)} \text{ iff } 2abd < b(ad + bc) = adb + b^2c
  \]
  but observe that $\cls{(a, b)} < \cls(c, d)$ means $ad < bc$ and thus
  \[
    2abd = abd + abd < abd + b^2c.
  \]
  Similarly,
  \[
    \cls{(ad + bc, 2bd)} < \cls{(c, d)} \text{ iff } d(ad + bc) = ad^2 + bcd < 2bcd
  \]
  which follows analogously from $ad < bc$.
\end{proof}

\newpage

\section{Lecture 2}

\emph{Material:} Section 2.2

\subsection{Reals (Dedekind's construction)}

\emph{Idea:} Real numbers `divide' the rationals uniquely into $(L, R)$

\begin{example}
  $\sqrt{2} = (\QQ_0^- \cup \{q \in \QQ^+ \mid q^2 < 2\}, \{q \in \QQ^+ \mid 2 < q^2 \})$.
\end{example}

\begin{observation}
  We always have for such $(L, R)$ that $R = \ol{L} := \QQ \setminus L$, hence the $R$ is redundant.
\end{observation}

\begin{definition}[Dedekind real]
  A Dedekind real number is $r \subseteq \QQ$ such that
  \begin{enumerate}[(i)]
  \item $\emptyset \subset r \subset \QQ$
  \item \emph{left-closed:} If $p \in r$ then $q \in r$ for all $q < p$
  \item \emph{no maximum:} If $p \in r$ then there is $q \in r$ with $p < q$
  \end{enumerate}
  We take $\RR := \{r \subseteq \QQ \mid r \text{ a Dedekind real}\}$
\end{definition}

\begin{example}
  For $p \in \QQ$ we embed it into $\RR$ as $p_\RR := \{q \in \QQ \mid q < p\}$
  \emph{not} $p \leq q$.

  Hence, for rational numbers, they end up in $\ol{p_\RR}$ rather than $p_\RR$.
  This is a choice.
\end{example}

\begin{proposition}
  $q_\RR$ is a Dedekind real.
\end{proposition}
\begin{proof}
  Everything but (iii) is obvious. For no maximum, let $p \in q_\RR$ then we
  know by density of $\QQ$ that there is $p < x < q$ for which $x \in q_\RR$ follows.
\end{proof}

\begin{fact}
  An $x \in \RR$ is $q_\RR$ for some $q \in \QQ$ iff $q \in \ol{x}$ is minimal.
\end{fact}

\begin{definition}
  We take $r =_\RR s$ iff $r = s$ for $r, s \in \RR$. Similarly, $r \leq_\RR s$
  iff $r \subseteq s$.
\end{definition}

\begin{fact}
  $r \leq_\RR s$ is a linear order.
\end{fact}

For $X \subseteq \RR, y \in \RR$ write $X \leq y$ if $x \leq y$ for all $x \in X$.

\begin{theorem}[Completeness]
  Let $\emptyset \neq A \subseteq \RR$ be such that $A \leq b$ for some $b \in \RR$.
  Then there exists $\sup A \in \RR$ such that $A \subseteq \sup A$ and for all
  $A \leq c$ we have $\sup A \leq c$. We call $\sup A$ the \emph{least upper
    bound} or \emph{supremum} of $A$.
\end{theorem}
\begin{proof}
  Simply take $\sup A := \bigcup A$. First, $\sup A \in \RR$:
  \begin{enumerate}[(i)]
  \item $\sup A \neq \emptyset$ because some $r \in A$. $\sup A \neq \QQ$
    because $\sup A \cap \ol{b} = \emptyset$.
  \item If $p \in \sup A$ then $p \in a \in A$ and thus $p > q \in a$ and hence
    $p \in \sup A$.
  \item Analogous to (ii).
  \end{enumerate}

  Now, the supremum property. Let $A \leq c$, then $\sup A \leq c$, i.e.\ $\sup
  A \subseteq c$. Let $p \in \sup A$ then $p \in a \in A$. As $a \leq c$ we thus
  know that $p \in c$.
\end{proof}

\begin{observation}
  $\QQ$ is famously incomplete. For example, $\sup \{q \in \QQ^+ \mid q^2 < 2\}
  \not\in \QQ$.
\end{observation}

\begin{definition}
  For $r, s \in \RR$ we define
  \begin{itemize}
  \item $r + s := \{p + q \mid p \in r, q \in s\}$
  \item $r \cdot s :=
    \begin{cases}
      \{m < p \cdot q \mid p \in \ol{r}, q \in \ol{s}\} & r \leq 0, s \leq 0 \\
      \{m < p \cdot q \mid p \in r, q \in \ol{s} \} & r \leq 0, s > 0 \\
      \{m < p \cdot q \mid p \in \ol{r}, q \in s \} & r > 0, s \leq 0 \\
      \{m < p \cdot q \mid p \in r_+, q \in s_+ \} \cup \QQ^-_0 & r > 0, s > 0 \\
    \end{cases}
    $
  \end{itemize}
\end{definition}

\begin{proposition}
  $\sqrt{2}^2 = 2$
\end{proposition}
\begin{proof}
  \begin{align*}
    \sqrt{2}^2 & = \QQ^-_0 \cup \{m < p \cdot q \mid 0 < p, q \in \sqrt{2}\} \\
               & = \QQ^-_0 \cup \{m < p \cdot q \mid 0 < p, q \wedge p^2 < 2, q^2 < 2\} \\
               & = \QQ^-_0 \cup \{m < p^2 \mid 0 < p\wedge p^2 < 2\} \\
               & = \QQ^-_0 \cup \{m < 2\} \\
               & = \{q < 2\} = 2_\RR
  \end{align*}
\end{proof}

\begin{proposition}
  There is no surjection $c : \NN \to \RR$.
\end{proposition}
\begin{proof}
  We construct a sequence $a_n < b_n$ such that $c(n) \not\in [a_{n + 1}, b_{n +
  1}]$. Let
  $a_0 = 0, b_n = 1$ then at each step $n$ consider, subdivide $[a_n, b_n]$ into
  $a_n, \frac 1 3 (a_n + b_n), \frac 2 3 (a_n + b_n), b$ and pick the subinterval not
  containing $c(n)$.

  Formally, fix
  \[
    x_0 = a_n \quad x_1 = \frac 1 3 (a_n + b_n) \quad x_2 = \frac 2 3 (a_n +
    b_n) \quad x_3 = b_n.
  \]
  We distinguish two cases:
  \begin{description}
  \item[$x_2 < c(n)$:] Then set $a_{n + 1} := x_0, b_{n + 1} := x_1$.
  \item[otherwise:] Let $i := \min \{0 \leq i \leq 2 \mid c(n) < x_i\}$ and set
    $a_{n + 1} := x_i, b_{n + 1} := x_{i + 1}$.
  \end{description}
  Observe that $a_n \leq a_{n + 1} < b_{n + 1} \leq b_n$ and $c(n) \not\in [a_{n
  + 1}, b_{n + 1}]$.

  Now consider $s := \sup \{a_n \mid n \in \NN\}$. Then $s \in [a_n, b_n]$ for
  all $n \in \omega$: Observe that $b_n \geq b_{n + k} > a_{n + k}$, i.e.\ $b_n$
  is an upper bound on the $a_n$, hence $s \leq b$ by completeness.

  As $c$ is a surjection, there is $k$ s.t.\ $s = c(k)$. But $c(k) \not\in [a_{k
  + 1},
  b_{k + 1}]$ whereas $s \in [a_{k + 1}, b_{k + 1}]$.
\end{proof}

% \newpage

% \section{Exercises}

% \begin{exercise}
%   Prove that multiplication in $\ZZ$ and $\QQ$ are well-defined, i.e. using a
%   congruent function.
% \end{exercise}
% \begin{solution}
%   \
%   \begin{description}
%   \item[$\QQ$:]
%     Let $(a, b) \sim (a', b')$ and $(c, d) \sim (c', d')$. Then $ab' = a'b$ and
%     $cd' = c'd$. Recall that
%     \[
%       \cls{(a, b)} \cdot_\QQ \cls{(c, d)} := \cls{(ac, bd)}.
%     \]
%     We want to show that $(ac, bd) \sim (a'c', b'd')$, i.e. that
%     $acb'd' = a'c'bd$. This follows directly from multiplying the previous equations.

%   \item[$\ZZ$:] Let $(a, b) \sim (a', b')$ and $(c, d) \sim (c', d')$. Then $a +
%     b' = a' + b$ and $c + d' = c' + d$. Recall that
%     \[
%       \cls{(a, b)} \cdot_\ZZ \cls{(c, d)} := \cls{(ac + bd,bc + ad)}.
%     \]
%     We want to show that $(ac + bd, bc + ad) \sim (a'c' + b'd', b'c' + a'd')$,
%     i.e. that
%     \[
%       ac + bd + b'c' + a'd' = a'c' + b'd' + bc + ad.
%     \]
%     Observe that
%     \[
%       a(c + d') + b(c' + d) = a(c' + d) + b(c + d') \text{ and } (a + b')c' +
%       (a' + b)d' = (a' + b)c' + (a + b')d'
%     \]
%     because $c + d' = c' + d$ and $a + b' = a' + b$. Adding these two equations
%     together yields exactly the desired equation (module canceling terms).

%     \emph{Idea:} $ab$-equation proves the claim only for $(c, d) \sim (c', d')$.
%   \end{description}
% \end{solution}

% \begin{exercise}[2.6]
%   Prove that $\RR$ is not bounded from above.
% \end{exercise}
% \begin{solution}
%   Suppose there was $x \in \RR$ such that $y \subseteq x$ for all $x \in \RR$.
%   This means for each $q \in \QQ$, $(q + 1)_\RR \subseteq x$ and thus $q \in x$.
%   But then $x = \QQ$, which means $x \not\in \RR$.
% \end{solution}

% \begin{exercise}[2.12]
%   Show that $r + 0 = r$.
% \end{exercise}
% \begin{solution}
%   Recall that $0 \in \RR$ is $\{q \in \QQ \mid q < 0 \}$. Then
%   \begin{align*}
%     r + 0 & = \{p + q \mid p \in r, q \in 0\} \\
%     &= \{p + q \mid p \in r, q < 0\}
%   \end{align*}
%   \begin{description}
%   \item[$r + 0 \subseteq r$:] For $p + q \in r + 0$ we know that $p + q < p$ and thus $p + q \in r$.
%   \item[$r \subseteq r + 0$:] Let $p \in r$, then there must be $p' > p \in r$.
%     Then $p - p' < 0$ and thus $p - p' \in 0$. Hence $p' + (p - p') = p \in r + 0$.
%   \end{description}
% \end{solution}

\printbibliography{}

\end{document}
