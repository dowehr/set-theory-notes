\documentclass{whrartcl}

\title{Set theory: Week 6}
\author{Dominik Wehr, University of Gothenburg}
\date{Updated: \today}

\usepackage[inference]{semantic}
\usepackage{mathpartir}
\usepackage{xcolor}
\usepackage{url}
\usepackage{bussproofs}

% Fix \mu in the bibliography
\DeclareUnicodeCharacter{03BC}{$\mu$}
\usepackage[
  style=numeric-comp,
  sorting=nyt,
  maxbibnames=99,
  backref=false,
  doi=false,
  isbn=false,
  url=false,
  eprint=false,
  backend=biber
  ]{biblatex}
\bibliography{bibliography}

\newcommand{\NN}{\mathbb{N}}
\newcommand{\ZZ}{\mathbb{Z}}
\newcommand{\QQ}{\mathbb{Q}}
\newcommand{\RR}{\mathbb{R}}
\newcommand{\BB}{\mathbb{B}}
\newcommand{\FF}{\mathcal{F}}
\newcommand{\PP}{\textbf{P}}
\newcommand{\DD}{\mathcal{D}}
\newcommand{\II}{\mathcal{I}}
\newcommand{\TT}{\mathcal{T}}
\newcommand{\Aa}{\mathcal{A}}
\newcommand{\Bb}{\mathcal{B}}
\newcommand{\Vv}{\mathcal{V}}
\newcommand{\WW}{\textbf{W}}
\newcommand{\LL}{\mathcal{L}}
\newcommand{\CC}{\mathcal{C}}
\newcommand{\pow}{\mathcal{P}}
\newcommand{\form}{\textsc{Form}}
\newcommand{\dom}{\text{dom}}
\newcommand{\ran}{\text{ran}}

\newcommand{\ol}[1]{\overline{#1}}
\newcommand{\abs}[1]{|#1|}
\newcommand{\uh}{\upharpoonright}
\newcommand{\irule}[3]{\inference[\textsc{#1}]{#2}{#3}}
\newcommand{\cls}[1]{\llbracket #1 \rrbracket}

\newcommand{\TODO}[1]{{\color{red}[TODO: #1]}}
\newcommand{\needcite}{{\color{red}[?]}}

\begin{document}

\maketitle

\section{Lecture 1}

\begin{remark}
  How do we count finite sets? Let $\{a, b, c, d\}$ be disjoint.
  \begin{itemize}
  \item $\{a, b\}$ has two elements. (Mark them)
  \item $\{a, b, c\}$ has three elements.
  \item $\{a, b, a\}$ has two elements
  \end{itemize}
\end{remark}

\begin{definition}
  A set $X$ is \emph{finite} if there is a bijection $f : n \to X$. A non-finite
  set is infinite.
\end{definition}

\emph{Question:} Can we also assign size to infinite sets? Spoiler: Not without
the AC, but we shall lay the groundwork.

\begin{definition}
  Let $A, B$ be sets.
  \begin{itemize}
  \item $A$ and $B$ are \emph{equinumerous} if there is a bijection $f : A \to
    B$. We write $A \simeq B$.
  \item $A$ is \emph{less than or equinumerous} with $B$ if there is an
    injection $f : A \to B$. We write $A \preceq B$.
  \item $A$ is \emph{dominated} by $B$ if $A \preceq B$ but $A \not\simeq B$ We
    write $A \prec B$.
  \end{itemize}
\end{definition}

\begin{lemma}
  Consider functions $f : A \to B, g : B \to C$ and $g \circ f := \{(a, g(f(a)))
  \mid a \in A\}$.
  \begin{enumerate}
  \item If $f, g$ are injections, so is $g \circ f$.
  \item If $f, g$ are surjections, so is $g \circ f$.
  \item If $f, g$ are bijections, so is $g \circ f$.
  \end{enumerate}
\end{lemma}

\begin{lemma}
  $\NN \simeq \ZZ$
\end{lemma}
\begin{proof}
  Consider the following bijection $b : \ZZ \to \NN$:
  \[
    b(z) :=
    \begin{cases}
      \abs{2z} & z \geq 0 \\
      \abs{-2z - 1} & z < 0
    \end{cases}
  \]
  \emph{Draw the image to illustrate.}
\end{proof}

\begin{lemma}
  For $n \in \NN$ we have $\NN \simeq \NN \setminus \{n\}$.
\end{lemma}
\begin{proof}
  For the bijection, consider
  \[
    b(m) :=
    \begin{cases}
      m & m < n \\
      m + 1 & m \geq n.
    \end{cases}
  \]
  \emph{Draw a picture again.}
\end{proof}

\begin{corollary}
  If $A \subseteq \NN$ is finite then $\NN \simeq \NN \setminus A$.
\end{corollary}

\begin{lemma}
  If $\NN \preceq X$ then $X$ is infinite.
\end{lemma}
\begin{proof}
  Suppose there was an injection $f : \NN \to X$ and a bijection $g : X \to n$
  for $n \in \NN$. This yields an injection $g \circ f$. We prove per induction
  on $n$ that no such injection can exist.
  \begin{description}
  \item[$n = 0$:] There can be no function $h : NN \to \emptyset$.
  \item[$n \to n + 1$:] Let $h : \NN \to n + 1$ be an injection. We know that a
    bijection $b : \NN \setminus \{h^{-1}(n)\} \to \NN$ exists. Observe $\ran(h
    \circ b) = (n + 1) \setminus \{n\} = n$ and that $h \circ b$ is an
    injection. But such an injection cannot exist per IH.
  \end{description}
\end{proof}

\begin{lemma}
  $\simeq$ is an equivalence relation.
\end{lemma}
\begin{proof}
  \
  \begin{description}
  \item[reflexive:] For a set $A$ the function $1_A := \{(a, a) \mid a \in A\}$
    is a bijection of the right type.
  \item[symmetric:] The inverse of a bijection is a bijection.
  \item[transitive:] Composing two bijections yields a bijection.
  \end{description}
\end{proof}

\begin{lemma}
  $\preceq$ is reflexive and transitive.
\end{lemma}
\begin{proof}
  \
  \begin{description}
  \item[reflexive:] $1_A$ is an injection.
  \item[transitive:] Composing injections yields an injection.
  \end{description}
\end{proof}

\begin{theorem}[Cantor-Schröder-Bernstein]
  $\preceq$ is anti-symmetric. I.e.\ if $A \preceq B$ and $B \preceq A$ then $A \simeq B$.
\end{theorem}
\begin{proof}
  \emph{Na\"ive Idea:} Start with the surjection $h^* : A \to B$
  \[
    h^*(a) :=
    \begin{cases}
      g^{-1}(a) & \text{ if this is defined} \\
      f(a) & \text{otherwise}
    \end{cases}.
  \]
  When can this violate injectivity? Whenever we have $g(f(a)) \neq a$ as both
  will map to $a$ \emph{(draw the triangle!)}. How to fix this? Redirect
  $g(f(a)) \mapsto f(g(f(a)))$. But this can cause further problems!

  Define
  \[
    A_0 := A \setminus \ran(g) \quad B_i := f(A_i) \quad A_{n + 1} := \{g(b)
    \mid b \in B_n\}
  \]
  \emph{Intuitively:} $a \in A_n$ for $n > 0$ if there is a problem with taking
  $a \mapsto g^{-1}(a)$ collision after $n$ redirection steps.

  Define $h : A \to B$
  \[
    h(a) :=
    \begin{cases}
      f(a) & \exists n.~a \in A_n \\
      g^{-1}(a) & \text{ otherwise}
    \end{cases}
  \]
  and observe that if $a \not\in A_0$ then $a \in \ran(g)$ so this is well-defined.

  \begin{description}
  \item[injectivity:] Let $h(a) = h(a')$. We distinguish which case of the
    definition applies.
    \begin{description}
    \item[$h(a) = f(a)$ and $h(a') = f(a')$:] Then $a = a'$ by injectivity of $f$.
    \item[$h(a) = g^{-1}(a)$ and $h(a') = g^{-1}(a')$:] Then $a = g(g^{-1}(a)) =
      g(g^{-1}(a')) = a'$.
    \item[$h(a) = f(a)$ and $h(a') = g^{-1}(a')$:] Then $a \in A_n$, $f(a) \in
      B_n$ and $g(f(a)) = a' \in B_{n+1}$, contradicting the defining condition
      for $h(a')$.
    \end{description}
  \item[surjectivity:] Let $b \in B$.
    \begin{description}
    \item[$b \in B_n$:] If $b \in B_n$ then there is an $a \in B_n$ with $h(a) = f(a) = b$.
    \item[$b \not\in B_n$ for all $n$:] Then $g(b) \not\in A_n$ for all $n \in
      \NN$: It cannot be in $A_0$. If $g(b) \in A_{n + 1}$ then $b \in B_n$,
      which we have ruled out.

      That means $h(g(b)) = g^{-1}(g(b)) = b$.
    \end{description}
  \end{description}
\end{proof}

\begin{lemma}
  $\prec$ is irreflexive and transitive.
\end{lemma}
\begin{proof}
  \
  \begin{description}
  \item[irreflexive:] For any set $A$, $A \simeq A$ via $1_A : A \to A$.
  \item[transitive:] Let $A \preceq B \preceq C$ with $A \not\simeq B \not\simeq
    C$. We showed before that $A \preceq C$. Suppose, towards contradiction,
    that $A \simeq C$. Then $B \preceq C \simeq A$ yields $B \preceq A$ (compose
    the underlying functions!). But by Cantor-Schröder-Bernstein, $A \preceq B$
    and $B \preceq A$ implies $A \simeq B$, contradicting our assumption.
  \end{description}
\end{proof}

\begin{corollary}
  If $A \preceq B \prec C$ or $A \prec B \preceq C$ we have $A \prec C$.
\end{corollary}


\begin{remark}
  We have shown that under ZL, $\preceq$ is a total relation.
\end{remark}


\newpage

\section{Lecture 2}

\subsection*{Countability}

\begin{definition}
  $A$ is \emph{countable} if it is finite or $A \simeq \NN$. We say that $A$ is
  countably infinite if $A \simeq \NN$. If a set is not countable, it is said to
  be \emph{uncountable}.
\end{definition}

\begin{theorem}
  If $A \subseteq N$ then either $A$ is finite or $A \simeq \NN$.
\end{theorem}
\begin{proofsketch}
  Observe that either $A \subseteq n \in \NN$ or for every $n \in A$ there is an
  $m \in A$ with $n < m$.
  \begin{description}
  \item[$A \subseteq n$:] We can prove per induction on $n$ that there exists an
    $m \leq n$ with $m \simeq n$.
  \item[Otherwise:] Consider $c : \Pi n \in A.~\{m \in A \mid n <
    m\}$ defined via
    \(
      c(n) := \min \{m \in A \mid n < m\}
      \). Then the following is an injection witnessing $\NN \preceq A$:
      \[
        i(0) := \min A \qquad \qquad f(n + 1) := c(f(n)).
      \]
  \end{description}
\end{proofsketch}

\begin{lemma}
  $\NN \times \NN \simeq \NN$
\end{lemma}
\begin{proof}
  Recall that $1 + \ldots + n = \frac{n(n + 1)}{2}$. Then consider the bijection
  $b : \NN \to \NN \times \NN$ with:
  \[b(n, m) := \frac{k(k + 1)}{2} + n \text{ where } k = n + m\]
  \emph{Draw the picture.}
\end{proof}
\begin{remark}
  The general statement that $A \simeq A \times A$ for infinite $A$ is
  equivalent to AC.
\end{remark}

\begin{corollary}
  $\NN \simeq \QQ$
\end{corollary}
\begin{proof}
   We use Cantor-Schröder-Bernstein. Clearly $\NN \preceq \QQ$. Recall also the bijection $b
   : \ZZ \to \NN$. Then we can define the $\preceq$-part of $\QQ \preceq \NN
   \times \NN \simeq \NN$ as follows:
   \[
     i(\cls{(a, b)}) := (b(z), b(z')) \text{ for } (z, z') \in \cls{(a, b)}
     \text{ with least } z'
   \]
\end{proof}

\begin{theorem}
  $\NN \simeq \ZZ \simeq \QQ$
\end{theorem}

\begin{lemma}
  Let $V$ be a countably infinite set of propositional variables and consider the set $\FF$
  of formulas generated by
  \[
    \varphi, \psi \in \FF ::= x \mid \neg \varphi \mid \varphi \wedge \psi \mid
    \varphi \vee \psi \qquad x \in V
  \]
  Then $\FF$ is countably infinite.
\end{lemma}
\begin{proof}
  We use CSB. Let $x \in V$. Then we obtain $\NN \preceq \FF$ via
  \[
    i(0) := x \qquad i(n + 1) := x \wedge i(n)
  \]
  Conversely, let $b : \NN \times \NN \to \NN$ and $v : V \to \NN$ be bijections. Then we obtain
  $\FF \preceq \NN$ via
  \begin{mathpar}
    j(x) := b(0, v(x))

    j(\neg \varphi) := b(1, j(\varphi))

    j(\varphi \wedge \psi) := b(2, b(j(\varphi), j(\psi)))

    j(\varphi \vee \psi) := b(3, b(j(\varphi), j(\psi))).
  \end{mathpar}
\end{proof}

\subsection*{Powerset and Reals}

\begin{corollary}
  $\NN \prec \RR$.
\end{corollary}
\begin{proof}
  We know that $\NN \preceq \RR$. We have also proven before that there can be no surjection
  $\NN \to \RR$ and thus that $\NN \not\simeq \RR$.
\end{proof}

\begin{lemma}
  $\pow(X) \simeq 2^X$
\end{lemma}
\begin{proof}
  Consider $\chi : \pow(X) \to 2^X$ given by
  \[
    \chi_A(x) :=
    \begin{cases}
      1 & x \in A \\
      0 & x \not\in A
    \end{cases}
  \]
\end{proof}

\begin{lemma}
  If $2 \preceq Y$ then $X \prec Y^X$.
\end{lemma}
\begin{proof}
  Observe that $X \preceq \pow(X) \simeq 2^X \preceq Y^X$. Now suppose $b : X \simeq
  Y^X$. Consider the function $f : X \to Y$ defined as follows:
  \[
    f(x) :=
    \begin{cases}
      i(1) & b(x)(x) = i(0) \\
      i(0) & \text{otherwise}
    \end{cases}
  \]
  and observe that $f(x) \neq b(x)(x)$ and thus $f \neq b(x)$ for every $x \in
  X$ and thus $f \not\in \ran(b)$. Then $b$ cannot be a surjection.
\end{proof}
\begin{corollary}
  $X \prec \pow{X}$
\end{corollary}

\begin{lemma}
  $\RR \simeq \pow(\QQ) \simeq 2^\NN \simeq \NN^\NN$.
\end{lemma}
\begin{proof}
  We use CSB. We already know $\RR \preceq \pow(\QQ) \simeq \pow(\NN) \simeq
  2^\NN$. It remains to find an injection $i : 2^\NN \to \RR$. Pick
  \[
    i(f) := \sup \left\{\left(\sum_{i = 0}^n \frac{f(i)}{10^i}\right)_\RR \mid n \in \NN\right\}
  \]
  Intuitively, this means $i(f) = f(0).f(1)f(2)f(3)\ldots$ as a sequence of
  digits.

  For $2^\NN \preceq \NN^\NN$ consider the following injection $b : \NN^\NN \to
  2^\NN$:
  \[
    b(f) := n \mapsto
    \begin{cases}
      0 & \exists k \in \NN.~n = \sum_{i = 0}^k (f(k) + 1) \\
      1 & \text{ otherwise }
    \end{cases}
  \]
  In other words,
  \[
    b(f) = 0 \underbrace{1 \ldots 1}_{f(0)} 0 \underbrace{1 \ldots 1}_{f(1)} 0
    \underbrace{1 \ldots 1}_{f(2)} 0 \ldots
  \]
  This is only an injection as $0^\omega \not\in \ran(b)$.
\end{proof}

\begin{lemma}
  $\RR \times \RR \simeq \RR$
\end{lemma}
\begin{proof}
  We show $2^\NN \times 2^\NN \simeq 2^\NN$ instead. Then simply consider
  \[
    b(f, g) := n \mapsto
    \begin{cases}
      f(k) & n = 2k \text{ for } k \in \NN \\
      g(k) & n = 2k + 1 \text{ for } k \in \NN
    \end{cases}
  \]
  i.e. we obtain the sequence $f(0)g(0)f(1)g(1)\ldots$.
\end{proof}

\begin{remark}
  The continuum hypothesis posits that there is no $A \subseteq \pow(\NN)$ such
  that $\NN \prec A \prec \pow(\NN)$. It is independent of ZFC!

  Contrast with the statement we proved earlier that every $A \subseteq \NN$ is countable.
\end{remark}

\printbibliography{}

\end{document}
