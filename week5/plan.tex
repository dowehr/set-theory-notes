\documentclass{whrartcl}

\title{Set theory: Week 5}
\author{Dominik Wehr, University of Gothenburg}
\date{Updated: \today}

\usepackage[inference]{semantic}
\usepackage{mathpartir}
\usepackage{xcolor}
\usepackage{url}
\usepackage{bussproofs}

% Fix \mu in the bibliography
\DeclareUnicodeCharacter{03BC}{$\mu$}
\usepackage[
  style=numeric-comp,
  sorting=nyt,
  maxbibnames=99,
  backref=false,
  doi=false,
  isbn=false,
  url=false,
  eprint=false,
  backend=biber
  ]{biblatex}
\bibliography{bibliography}

\newcommand{\NN}{\mathbb{N}}
\newcommand{\ZZ}{\mathbb{Z}}
\newcommand{\QQ}{\mathbb{Q}}
\newcommand{\RR}{\mathbb{R}}
\newcommand{\BB}{\mathbb{B}}
\newcommand{\PP}{\textbf{P}}
\newcommand{\DD}{\mathcal{D}}
\newcommand{\II}{\mathcal{I}}
\newcommand{\TT}{\mathcal{T}}
\newcommand{\FF}{\mathcal{F}}
\newcommand{\Aa}{\mathcal{A}}
\newcommand{\Bb}{\mathcal{B}}
\newcommand{\Vv}{\mathcal{V}}
\newcommand{\WW}{\textbf{W}}
\newcommand{\LL}{\mathcal{L}}
\newcommand{\CC}{\mathcal{C}}
\newcommand{\pow}{\mathcal{P}}
\newcommand{\form}{\textsc{Form}}
\newcommand{\dom}{\text{dom}}
\newcommand{\ran}{\text{ran}}

\newcommand{\ol}[1]{\overline{#1}}
\newcommand{\abs}[1]{|#1|}
\newcommand{\uh}{\upharpoonright}
\newcommand{\irule}[3]{\inference[\textsc{#1}]{#2}{#3}}
\newcommand{\cls}[1]{\llbracket #1 \rrbracket}

\newcommand{\TODO}[1]{{\color{red}[TODO: #1]}}
\newcommand{\needcite}{{\color{red}[?]}}

\begin{document}

\maketitle

\section{Lecture 1}

\emph{Material:} Section 5.2

\begin{definition}
  The \emph{axiom of choice} \textsc{AC} states the following: Let $\FF$ be a set such that
  for all $A \in \FF$ we have $A \neq \emptyset$. Then there exists a function
  $c : \FF \to \bigcup \FF$ such that for all $A \in \FF$, $c(A) \in A$. We call
  $c$ a \emph{choice function}.
\end{definition}

\begin{lemma}[Inverting surjections]
  Let $f : X \to Y$ be a surjection. Then there exists an injection $g : Y \to
  X$ such that $f(g(y)) = y$ for all $y \in Y$.
\end{lemma}
\begin{proof}
  For $x, x' \in X$ we set $x \sim x'$ iff $f(x) = f(x')$. This is an
  equivalence relation. Setting $\FF = X_{/ \sim}$, there exists a choice
  function $c : X_{/ \sim} \to X$ such that for each $\cls{x} \in X_{/ \sim}$ we
  have $c(\cls{x}) \in \cls{x}$. Now define
  \[
    g(y) \coloneq c(\{x \in X \mid f(x) = y\}).
  \]
  It is clear that $f(g(y)) = y$. For injectivity, let $y, y'$ be such that
  $g(y) = g(y')$. But then $y = f(g(y)) = f(g(y')) = y'$.
\end{proof}

\begin{remark}
  We will soon show that AC is actually required to prove this.
\end{remark}

\begin{definition}
  We define two more forms of the AC:
  \begin{description}
  \item[Disjoint form:] Let $\FF$ be such that for each $A, A' \in \FF$, $A \neq
    \emptyset$ and if $A \neq A'$ then $A \cap A' = \emptyset$. Then there
    exists a choice function $c : \FF \to \bigcup \FF$.
  \item[Power set form:] Fix a set $A \neq \emptyset$. Then there exists a
    choice function $c : (\pow(A) \setminus \{\emptyset\}) \to A$.
  \end{description}
\end{definition}

\begin{theorem}
  The three forms of AC are equivalent.
\end{theorem}
\begin{proof}
  \
  \begin{description}
  \item[AC to Power set:] Simply observe that $\pow{A} \setminus \{\emptyset\}$
    is a suitable family for AC.
  \item[Power set to Disjoint:] Let $\FF$ be a disjoint family. Observe that
    $\FF \subseteq \pow(\bigcup \FF) \setminus \{\emptyset\}$, meaning we can
    obtain a choice function on $\FF$ by restricting that on $\pow(\bigcup \FF) \setminus \{\emptyset\}$.
  \item[Disjoint to AC:] Let $\FF$ be an arbitrary family with $A \neq
    \emptyset$ for all $A \not \FF$. Using replacement, construct
    \[
      \FF' := \{A \times \{A\} \mid A \in \FF\}.
    \]
    Then $\FF'$ is a disjoint family: Let $A \times \{A\}, A' \times \{A\} \in
    \FF'$ be distinct. Suppose there was $\langle a, B \rangle \in A \times
    \{A\} \cap A' \times \{A'\}$. By construction, $A = B = A'$, contradicting
    their distinctness.

    Now we simply obtain the desired choice function via
    \[
      \FF \ni A \mapsto A \times \{A\} \in \FF' \xrightarrow{c} \bigcup \FF' \ni
      \langle a, A \rangle \mapsto a \in \bigcup \FF.
    \]
  \end{description}
\end{proof}

\begin{lemma}
  Invertability of surjections implies AC.
\end{lemma}
\begin{proof}
  Suppose every surjection $f : X \to Y$ had an inverse $g : Y \to X$ with $y =
  f(g(y))$ for all $y \in Y$. We prove disjoint AC: Let $\FF$ be disjoint. Then
  the following is a surjection
  \[
    f := \bigcup \FF \ni a \mapsto A \in \FF \text{ s.t. } a \in A.
  \]
  Then it has an inverse $g : \FF \to \bigcup \FF$ such that $f(g(A)) = A$, i.e.
  $g(A) \in A$. This is the desired choice function.
\end{proof}

\begin{description}
\item[Desirable consequences:] 
  \
  \begin{enumerate}
  \item Logic: Completion into of maximal consistent theories
    (~Completeness and Compactness)
  \item CT: Any full, faithful, essentially surjective functor is an equivalence
  \item Linear algebra: Every vector space has a basis $\star$
  \item Algebra: Krull's theorem
  \item Topology: Products of compact spaces are compact
  \end{enumerate}
  \
\item[Dubious consequences:]
  \
  \begin{enumerate}
  \item Intuitionistically: LEM
  \item Banach-Tarski: A ball in $\RR^3$ can be partitioned into finitely many
    parts which, using translations and rotations, can be recombined into two balls.
  \end{enumerate}
\end{description}


\begin{definition}
  We define another form of the AC:
  \begin{description}
  \item[Index form:] Let $I$ be a set and $A$ a function with $\dom(A) = I$ such
    that $A_i \neq \emptyset$ for all $i \in I$.
    Then there exists a function $c : \Pi i \in I.~A_i$, i.e. a function $c : I
    \to \bigcup_{i \in I} A_i$ such that $c(i) \in A_i$ for all $i \in I$.
  \end{description}
\end{definition}

\begin{exercise}
  The index form and plain AC are equivalent.
\end{exercise}

\begin{lemma}
  Let $n \in \NN$ and let $A$ be a function with $\dom(A) = n$ and $A_i \neq
  \emptyset$ for all $i < n$. Then there
  exists $c : \Pi i < n.~A_i$.
\end{lemma}
\begin{proof}
  Induction on $n$.
  \begin{description}
  \item[$n = 0:$] Then $n = \emptyset$, thus $c = \emptyset$ is the desired function.
  \item[$n + 1:$] Suppose there was $c' : \Pi i < n.~A_i$. We know that $A_n \neq
    \emptyset$ and thus there is an $a_n \in A_n$. Then
    \[
      c := c' \cup \{(n, a_n)\} : \Pi i < n + 1.~A_i
    \]
    exists as desired.
  \end{description}
\end{proof}

\begin{remark}
  The aforementioned theorem already fails for $I = \NN$. Crucially,
  \[
    c := \bigcup \{c_i \mid i \in \NN\} \text{ for sequence } c_i \subseteq c_{i +
    1} : \Pi j < i + 1.~A_j
  \]
  fails generally, because we cannot pick \emph{unique} $c_i$ to construct such
  a sequence. (Compare recursion theorem)
\end{remark}


\newpage

\section{Lecture 2}

\emph{Material:} Section 5.4

\begin{theorem}
  There is a function $c : \pow(\NN) \setminus \{\emptyset\} \to \NN$.
\end{theorem}
\begin{proof}
  Simply take $c(A) := \text{least } n \in A$. As $\NN$ is well-ordered by
  $\leq$ this always exists and is unique.
\end{proof}


\begin{definition}
  The \emph{well-ordering principle} WOP states: For every set $X$ there exists
  a relation $\prec\; \subseteq X \times X$ which well-orders $X$.
\end{definition}

\begin{corollary}
  The WOP implies AC.
\end{corollary}


\begin{definition}
  A relation $\preceq\; \subseteq X \times X$ is a \emph{partial order} if it is
  reflexive, antisymmetric and transitive. A set $Y \subseteq X$ is a
  \emph{$\preceq$-chain} if $\preceq$ is total on $X$. An element $x \in X$ is
  \emph{maximal} if there is no $x \prec x'$.
\end{definition}

\begin{definition}
  \emph{Zorn's lemma} ZL states: Let $X$ be partially ordered by $\preceq\;
  \subseteq X \times X$ in such a way that every $\preceq$-chain $Y \subseteq X$
  has an upper bound in $X$. Then $X$ has a $\preceq$-maximal element.
\end{definition}

\begin{theorem}
  The AC implies Zorn's lemma.
\end{theorem}
\begin{remark}
  We will give a simple proof of this at the end of the course.
\end{remark}
% \begin{proof}
%   \TODO{Fix me!}
%
%   Suppose $X$ had no maximal element. That means for each chain $Y \subseteq X$
%   there exists a strict upper bound $Y \prec y$: We know there is $Y \preceq
%   y'$, an upper bound on $Y$. If there was no $y' \prec y$, $y'$ would be
%   maximal. Hence, consider the following function
%   \[
%     c : \Pi Y \in \{Y \subseteq X \mid Y \text{ a $\prec$-chain}\}.~\{y \in X
%     \mid Y \prec y\}
%   \]
%   which chooses such a strict upper bound for each chain. Let $Y \subseteq X$ be
%   a chain and $x \in Y$ then we write $Y \uh x := \{y \in Y \mid y < x\}$ for
%   its initial segment in $Y$. We say a chain $A \subseteq X$ is a $c$-chain if:
%   \begin{enumerate}
%   \item $A$ is well-ordered by $\prec$
%   \item if for every $a \in A$ we have $a = c(A \uh a)$.
%   \end{enumerate}

%   Now consider
%   \[
%     C := \{A \subseteq X \mid A \text{ is a $c$-chain}\} \text{  and  } U :=
%     \bigcup C
%   \]
%   \begin{enumerate}
%   \item If $A, B \in C$ distinct then $B = A \uh a$ for $a \in A$ or $A = B \uh
%     b$ for $b \in B$: Let w.l.o.g.\ $b \in B \setminus A$ be minimal. Then $A =
%     B \uh b$. Suppose not, then let $a \in A \setminus (B \uh b)$ be minimal.
%   \item If $A \in C$ and $a \in A$ then $U \uh a = A \uh a$: Only $U \uh a
%     \subseteq A \uh a$ needs to be argued. Let $b \prec a$
%     with $b \in U$. Then $b \in B \in C$. We distinguish two cases:
%     \begin{description}
%     \item[$A = B \uh c$:] Then $a \preceq c$ as $a \in A$ and $b \in A$ as $b
%       \prec a \preceq c$ and thus $b \in A \uh a$.
%     \item[$B = A \uh c$:] Then $b \in A \uh a$. \TODO{Clean}
%     \end{description}
%   \item $U$ is a chain: Let $y, z \in U$. Then there are $c$-chains $(y \in Y),
%     (z \in Z) \in U$. Let w.l.o.g.\ $Y \subseteq Z$ then $y, z \in Z$ and thus
%     $y \prec z$ or $y = z$ or $z \prec y$.
%   \item $U$ is well-ordered by $\prec$: Let $Y \subseteq U$ and $a \in Y$. Then
%     $a \in A \in C$. We know that $U \uh a \subseteq A$ and thus $Y' := (U \uh a \cup \{a\})
%     \cap Y \subseteq A$. As $A$ is well-ordered by $\prec$, there is $b \in Y'$
%     with $b \preceq Y'$. Then $b \preceq Y$: Let $c \in Y \setminus Y'$ then $b
%     \preceq a \prec c$.
%   \item For $a \in U$, $a = c(U \uh a)$. Observe that $a \in A \in C$ and $a =
%     c(A \uh a) = c(U \uh a)$.
%   \end{enumerate}
%   Hence, $U \in C$. Consider $u = c(U)$ and observe that $U < u$ and $U \cup
%   \{u\} \in C$. But then $x \in C$ and thus $x \in U$. But this contradicts $U <
%   u$.
% \end{proof}

\begin{theorem}
  Zorn's lemma implies WOP.
\end{theorem}
\begin{proof}
  Let $X$ be a set. For $R \subseteq X \times X$ define the \emph{support of
    $R$} as $\sup(R) := \{x \in X \mid \exists y \in X.~(x, y) \in R \vee (y, x)
  \in R\}$. Consider the set
  \[
    W = \{R \subseteq X \times X \mid R \text{ well-orders } \sup(R)\}.
  \]
  Then $W$ is partially ordered by $R \sqsubseteq R'$: $R \subseteq R'$ and for
  every $x \in \sup(R)$ and $y \in \sup(R') \setminus \sup(R)$ we have $(x, y)
  \in R'$.

  We claim that every $\sqsubseteq$-chain in $W$ has an upper bound: Let $U \subseteq W$ be
  a $\sqsubseteq$-chain. First of all, $V := \bigcup U$ is in $W$.
  \begin{description}
  \item[irreflexive:] Suppose $(x, x) \in V$ then $(x, x) \in R \in U$
    but that entails $R \not\in W$.
  \item[transitive:] Suppose $(x, y), (y, z) \in V$ then $(x, y) \in R,
    (y, z) \in R'$ with $R, R' \in U$. As $U$ is a $\sqsubseteq$-chain, w.l.o.g. $R \subseteq
    R'$ and thus $(x, y) \in R'$, yielding $(x, z) \in R'$ and thus $(x, z) \in
    V$.
  \item[linear:] Let $x, y \in \sup(V)$ then $x \in \sup(R), y \in
    \sup(R')$ for $R, R' \in U$. Again, w.l.o.g.\ $R \subseteq R'$ by
    $\sqsubseteq$-linearity of $U$ and thus $x, y \in \sup(R')$ meaning $(x, y) \in
    R'$ or $(y, x) \in R'$ which transfer to $V$.
  \item[well-order:] Let $A \subseteq \sup(V)$ such that $x \in A$. Then
    $x \in \sup(R)$ for $R \in U$. Then $A' := \{y \in A \mid (y, x) \in V \} \subseteq
    \sup(R)$: Suppose not, i.e.\ suppose there was $z \in \sup(V) \setminus \sup(R)$ such that
    $(z, x) \in V$. There must be $R' \in U$ such that $(z, x) \in R'$. As $z
    \not\in \sup(R)$ we know that $R \sqsubseteq R'$ by $\sqsubseteq$-linearity.
    But that would entail $(x, z) \in R'$ and thus $(x, x) \in R'$,
    contradicting the linearity of $R'$.

    Hence $A' := \{y \in A \mid (y, x) \in V\} \subseteq \sup(R)$. Let $m :=
    \min_R A'$ then we claim $m = \min_V A$. Let $z \in A \setminus A'$ we know
    that $m V x V z$.
  \end{description}
  Now we claim that $V$ is a $\sqsubseteq$-upper bound on $U$: Let $R \in U$,
  clearly $R \subseteq U$. Now suppose $x \in \sup(R)$ and $y \in \sup(U)
  \setminus \sup(R)$ then there must be $R' \sqsupseteq R$ in $U$ such that $y
  \in \sup(R')$ and thus $(x, y) \in R' \subseteq V$.

  Then there must be a $\sqsubseteq$-maximal element $R$ of $W$. Suppose there
  was $x \in X \setminus \sup(R)$, then we can obtain a $R' \sqsupset R$ by taking
  \[R' := R \bigcup \{(y, x) \mid y \in \sup(R)\}.\]
  \emph{Exercise:} $R' \in W$. Thus $R$ could not have been maximal after all.
\end{proof}

\begin{remark}
  This ``closes'' the circle AC iff ZL iff WPO.
\end{remark}

\begin{lemma}
  Under Zorn's lemma, there is an injection $A \to B$ or an injection $B \to A$.
\end{lemma}
\begin{proof}
  Consider
  \[
    I := \{f \subseteq A \times B \mid f \text{ an injection} \}
  \]
  which is pre-ordered by $\subseteq$. Now let $C \subseteq I$ be a
  $\subseteq$-chain, then $\bigcup C$ is an injection (\emph{Exercise}!).
  By Zorn's lemma, there is a maximal $f \in I$. We distinguish three cases:
  \begin{description}
  \item[$\dom(f) = A$:] Then $f : A \to B$ is an injection as desired.
  \item[$\ran(f) = B$:] Then $f^{-1} := \{(b, a) \mid (a, b) \in f\}$ is an
    injection $f^{-1} : B \to A$. (Observe that $f$ is functional iff $f^{-1}$
    is injective).
  \item[$\dom(f) \subset A$ and $\ran(f) \subset A$:] Then there are $a \in A
    \setminus \dom(f)$ and $b \in B \setminus \ran(f)$. Then
    \[
      f' := f \bigcup \{(a, b)\}
    \]
    is an injection with $f \subset f'$, so $f$ was not maximal after all.
  \end{description}
\end{proof}


% \newpage

% \section{Exercises}

% \begin{exercise}
%   The index form and plain AC are equivalent.
% \end{exercise}
% \begin{solution}
%   \begin{description}
%   \item[Index to AC:] Simply take $I := \FF$ and $A_{B \in \FF} := B$.
%   \item[AC to Index:] Take $\FF := \{A_i \mid i \in I\}$, yielding a choice function $c' : \FF \to
%     \bigcup \FF$. Then take $c(i) := c(A_i) : \Pi i \in I.~A_i$.
%   \end{description}
% \end{solution}

% \begin{exercise}[5.10]
%   The \emph{invertability of injections} states that if $f : X \to Y$ with $X
%   \neq \emptyset$ is an
%   injection, there is a sujrection $g : Y \to X$ such that $g(f(x)) = x$ for all
%   $x \in X$.

%   Does this principle require AC? Is it equivalent to AC?
% \end{exercise}
% \begin{solution}
%   It does not require AC. Fix $b \in X$ then we define
%   \[
%     g(y) :=
%     \begin{cases}
%       x & y = f(x) \text{ for some } x \in X \\
%       b & \text{otherwise}
%     \end{cases}.
%   \]
%   Let $x \in X$ then $g(f(x)) = x$ because $f$ is an injection, i.e. there is no
%   $x' \neq x$ with $f(x') = f(x)$. For surjectivity, simply observe again that
%   $g(f(x)) = x$ and thus $\ran(g) = X$.
% \end{solution}

% \begin{exercise}[5.11]
%   The AC is equivalent to: For every $R \subseteq A \times B$ there exists a
%   function $f : \dom(R) \to B$ with $f \subseteq R$.
% \end{exercise}
% \begin{proof}
%   \begin{description}
%   \item[AC to Statement:] Use the index form with $I = \dom(R)$ and $A_a := \{b
%     \in B \mid (a, b) \in R\}$.
%   \item[Statement to AC:] Fix $\FF$ as usual. Then define the relation as
%     follows:
%     \[
%       R := \{
%       (F, f) \in \FF \times \bigcup \FF \mid f \in F
%       \}
%     \]
%     We obtain a function $c : \FF \to \bigcup \FF$ such that $c \subseteq R$,
%     meaning $c(F) \in F$ for all $F \in \FF$.
%   \end{description}
% \end{proof}

% \begin{exercise}[5.24]
%   Zorn's lemma implies AC.
% \end{exercise}
% \begin{solution}
%   Observe that, by the lecture, ZL $\to$ WPO $\to$ AC. We give a direct proof.

%   We prove the equivalent statement of 5.11: For $R \subseteq A \times B$ there
%   exists $f : \dom(R) \to B$. Consider the set
%   \[
%     P = \{f \subseteq A \times B \mid f \text{ a function and } f \subseteq R\}
%   \]
%   which is partially ordered by $\subseteq$. We have shown before that this is
%   closed under chains. Hence there is a maximal $f \in P$. If $\dom(f) \neq
%   \dom(R)$, we could extend $f$, contradicting maximality. Hence $f : \dom(R)
%   \to B$.
% \end{solution}

\printbibliography{}

\end{document}
