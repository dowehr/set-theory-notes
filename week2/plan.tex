\documentclass{whrartcl}

\title{Set theory: Week 2}
\author{Dominik Wehr, University of Gothenburg}
\date{Updated: \today}

\usepackage[inference]{semantic}
\usepackage{mathpartir}
\usepackage{xcolor}
\usepackage{url}
\usepackage{bussproofs}

% Fix \mu in the bibliography
\DeclareUnicodeCharacter{03BC}{$\mu$}
\usepackage[
  style=numeric-comp,
  sorting=nyt,
  maxbibnames=99,
  backref=false,
  doi=false,
  isbn=false,
  url=false,
  eprint=false,
  backend=biber
  ]{biblatex}
\bibliography{bibliography}

\newcommand{\NN}{\mathbb{N}}
\newcommand{\RR}{\mathcal{R}}
\newcommand{\BB}{\mathbb{B}}
\newcommand{\PP}{\textbf{P}}
\newcommand{\DD}{\mathcal{D}}
\newcommand{\II}{\mathcal{I}}
\newcommand{\TT}{\mathcal{T}}
\newcommand{\Aa}{\mathcal{A}}
\newcommand{\Bb}{\mathcal{B}}
\newcommand{\Vv}{\mathcal{V}}
\newcommand{\WW}{\textbf{W}}
\newcommand{\LL}{\mathcal{L}}
\newcommand{\CC}{\mathcal{C}}
\newcommand{\pow}{\mathcal{P}}
\newcommand{\form}{\textsc{Form}}
\newcommand{\dom}{\text{dom}}
\newcommand{\ran}{\text{ran}}

\newcommand{\ol}[1]{\overline{#1}}
\newcommand{\abs}[1]{|#1|}
\newcommand{\uh}{\upharpoonright}
\newcommand{\irule}[3]{\inference[\textsc{#1}]{#2}{#3}}

\newcommand{\TODO}[1]{{\color{red}[TODO: #1]}}
\newcommand{\needcite}{{\color{red}[?]}}

\begin{document}

\maketitle


\section{Lecture 1}

\emph{Material:}  ZF4--5, Cartesian products, Functions \& Relations, Natural
numbers (Sections 4.3, Section 3.1)

\emph{Book Recommendation:} How to Prove It: A Structured Approach, Daniel J. Velleman


\subsection{Zermelo-Fraenkel: Axioms 4--5}

\emph{Reading:} Section 4.4

\begin{definition}
  We fix the following axioms:
  \begin{description}
  \item[ZF4:] Separation

    $\forall p_1 \ldots \forall p_n \forall x \exists y \forall z.~z \in y \leftrightarrow (z \in x \wedge
    A(z, \ol{p}))$ where $A(-)$ is any formula with at most $n + 1$ free variables

    For any set $x$ and parameters $\ol{p}$, the set $\{z \in x \mid A(z, p)\}$ exists

  \item[ZF5:] Power set

    $\forall x \exists y \forall z.~z \in y \leftrightarrow z \subseteq x$

    The power set $\pow(x)$ of $x$ exists
  \end{description}
\end{definition}

\begin{proposition}[Intersection, Difference]
  If $x, y$ are sets then so are $x \cap y$ and $x \setminus y$.
\end{proposition}
\begin{proof}
  \
  \begin{description}
  \item[$x \cap y$:] Observe that $x \cap y = \{z \in x \mid z \in y\}$.
    Formally, this uses the parameter $y$.
  \item[$x \setminus y$:] Observe that $x \setminus y = \{z \in x \mid z \not\in
    y\}$.
  \end{description}
\end{proof}

\begin{proposition}[Cartesian Products]
  For sets $A, B$, $A \times B = \{\langle a, b \rangle \mid a \in A, b \in B\}$
  is a set.
\end{proposition}
\begin{proof}
  Observe that for $a \in A, b \in B$
  \[
    \langle a, b \rangle = \{\{a\}, \{a, b\}\} \in \pow(\pow(A \cup B))
  \]
  thus set $A \times B := \{p \in \pow(\pow(A \cup B))\mid \exists a
  \exists b.~p = \langle a, b \rangle\}$.
\end{proof}

\begin{definition}
  A relation between sets $A$ and $B$ is an $R \subseteq A \times B$. For any
  relation $R \subseteq A \times B$, we define its \emph{domain} as $\dom(R)
  \coloneq \{a \in A \mid \exists b \in B.~\langle a, b \rangle \in R\}$ and
  conversely its \emph{range} as $\ran(R) \coloneq \{b \in B \mid \exists a \in
  A.~ \langle a, b \rangle \in R\}$.

  A relation $R \subseteq A \times B$ is:
  \begin{itemize}
  \item \emph{total:} if for every $a \in A$ there is $b \in B$ s.t.\ $\langle
    a, b \rangle \in R$
  \item \emph{functional:} if for every $\langle a, b \rangle, \langle c, d
    \rangle \in R$ if $a = c$ then $b = d$
  \item \emph{injective:} if for every $\langle a, b \rangle, \langle c, d
    \rangle \in R$ if $b = d$ then $a = c$
  \item \emph{surjective:} if for every $b \in B$ there is $a \in A$ s.t.\
    $\langle a, b \rangle \in R$
  \end{itemize}
  We call a total functional relation a \emph{function}. We define $B^A \coloneq
  \{f \subseteq A \times B \mid f \text{ a function}\}$ and write $f : A \to B$
  to mean $f \in B^A$. For $f : A \to B$ and $a \in A$ we write $f(a)$ to mean
  the unique $b \in B$ s.t.\ $\langle a, b \rangle \in f$.
  We call $f : A \to B$ a \emph{bijection} if it is injective and surjective.
\end{definition}

\begin{proposition}
  For $f, g : A \to B$ we have $f = g$ iff for all $a \in A$, $f(a) = g(a)$.
\end{proposition}
\begin{proof}
  Only backwards is tricky. Let $\langle a, f(a) \rangle \in f$ then, as $f(a) =
  g(a)$, we have $\langle a, f(a) \rangle \in g$.
\end{proof}


\subsection{Natural numbers}

\begin{definition}[Natural numbers]
  Fix a set $N$. We call it a natural number set if it comes equipped with:
  \begin{itemize}
  \item an element $0 \in N$
  \item a function $s : N \to N$
  \end{itemize}
  with the following properties
  \begin{itemize}
  \item $s : N \to N$ is injective
  \item non-looping: for every $x \in N$, $s(x) \neq 0$
  \item inductive: for every set $P \subseteq N$ with $0 \in P$ and for every $p \in P$,
    $s(p) \in P$ we have $P = N$.
  \end{itemize}
\end{definition}

\begin{proposition}
  Let $N$ be a natural number set. Then $s(x) \neq x$ for every $x \in N$.
\end{proposition}
\begin{proof}
  We use inductivity to show $P := \{n \in N \mid s(n) \neq n\} = N$. We know
  that $0 \in P$ per assumption about $N$. Suppose that $n \in P$, i.e.\ $s(n) \neq n$, and
  assume $s(s(n)) = s(n)$. By injectivity this would mean $s(n) = n$, a
  contradiction, hence $s(n) \in P$.
\end{proof}

\begin{observation}
  With the axioms so far, we cannot prove that infinite sets exist. However, any
  natural number set must be infinite. Hence, we cannot yet prove that such a
  set exists.
\end{observation}

\newpage
\section{Lecture 2}

\emph{Material}: Natural numbers, Axiom of Infinity, Recursion (Sections 3.1 - 3.2, 3.4, Section 4.5)


\subsection{Natural numbers, cont'd}

\begin{remark}[Constructing $\NN$]
  A natural choice is to take $0 := \emptyset$. There are various ways to define $s(n)$:
  \begin{itemize}
  \item $s(n) = \{n\} \leadsto \emptyset, \{\emptyset\}, \{\{\emptyset\}\},
    \ldots$ (count set-levels)
  \item $s(n) = \langle \emptyset, n \rangle \leadsto \emptyset, \langle
    \emptyset \rangle, \langle \emptyset, \emptyset \rangle, \ldots$ (count
    tuple size)
  \item $s(n) = n \cup \{n\} \leadsto \emptyset, \{\emptyset\}, \{\emptyset,
    \{\emptyset\}\}, \ldots$ (count elements)
  \end{itemize}
  We shall pick the last option, but every variant would work.
\end{remark}

\begin{definition}
  A set $X$ is called \emph{inductive} if $\emptyset \in X$ and whenever $x \in
  X$ then $x \cup \{x\} \in X$.

  We fix the following additional axiom:
  \begin{description}
  \item[ZF7:] Infinity

    $\exists x.~\emptyset \in x \wedge \forall y \in x.~y \cup \{y\} \in x$

    There exists an inductive set.
  \end{description}
\end{definition}

\begin{remark}
  The set given by ZF7 need not be $\NN = \{\emptyset, \{\emptyset\}, \ldots\}$.
  For example, consider the set
  \[
    \NN \cup \{\{\{\emptyset\}\}, \{\{\emptyset\}, \{\{\{\emptyset\}\}\}\}, \ldots\}
  \]
\end{remark}

\begin{definition}
  Let $I$ be the inductive set given by ZF7. Then we define
  \[\NN \coloneq \bigcap \{J \subseteq I \mid J \text{ is inductive}\}.\]
\end{definition}

\begin{observation}
  By construction, every $n \in \NN$ is such that its members are its
  predecessors $n = \{0, \ldots, n-1\}$. In other words, $m < n$ iff $m \in n$.
\end{observation}

\begin{definition}
  We write $m < n$ to mean $m \in n$. Relatedly, $m \leq n$ is $m \in n$ or $m =
  n$.
\end{definition}

\begin{proposition}
  The relation $< \subseteq \NN \times \NN$ is a linear order:
  \begin{itemize}
  \item transitive: $m < n$ and $n < l$ entails $m < l$
  \item irreflexive: $n \not< n$
  \item linear: $m < n$ or $m = n$ or $n < m$
  \end{itemize}
\end{proposition}
\begin{proof}
  \
  \begin{itemize}
  \item transitive: Fix $m$ and $n$ and prove that
    \[
      P := \{l \in \NN \mid n < l \to m < l\} = \NN
    \]
    by induction. For $0$ this is trivial as $n < 0$ is impossible. For the
    successor case, let $n \in l \cup \{l\}$: If $n \in l$, then $m \in l$ per
    IH and thus $m \in l \cup \{l\}$. If $n = l$ then $m \in n = l \subseteq l
    \cup \{l\}$.
  \item irreflexive: Again, proof per induction. For $0$ this is trivial. Now
    suppose $n \cup \{n\} \in n \cup \{n\}$.
    \begin{description}
    \item[$s(n) \in n$:] Then we have $n \in s(n) \in n$ and thus $n \in n$, a
      contradiction by IH.
    \item[$s(n) = n$:] We have disproved this earlier.
    \end{description}
  \item linearity: Exercise for you!
  \end{itemize}
\end{proof}

% \begin{lemma}
%   If $m < n$ then $s(m) < s(n)$.
% \end{lemma}
% \begin{proof}
%   Proof per induction that
%   \[
%     \{n \mid \forall m.~m < n \to s(m) < s(n) \} = \NN
%   \]
%   on $n$. Trivial in the $0$ case. Let $m < s(n)$ then:
%   \begin{description}
%   \item[$m < n$:] Then $s(m) < s(n) < s(s(n))$.
%   \item[$m = n$:] Then $s(m) = s(n) < s(s(n))$.
%   \end{description}
% \end{proof}

\begin{proposition}
  The set $\NN$ is a natural number set.
\end{proposition}
\begin{proof}
  Choose $0 \coloneq \emptyset$ and observe that $\emptyset \in \NN$ because
  $\emptyset$ is in every inductive set. For $s : \NN \to \NN$ choose $s(n) := n
  \cup \{n\}$ and for well-definedness let $n \in \NN$ and thus $n \in J$ for
  every inductive set. Then, as such $J$s are inductive, $n \cup \{n\} \in J$ as
  well and thus $s(n) \in \NN$.

  Inductive: Let $P \subseteq N$ be such that $\emptyset \in P$ and for $p \in
  P$, $s(p) \in P$ as well. Then $P$ is inductive and hence $N \subseteq P$.

  Non-looping: Observe that for any $n$, $n \cup \{n\} \neq \emptyset$ because
  $n \in n \cup \{n\}$.

  Injection: Left as an exercise/read the book!
\end{proof}

\subsection{Finite and countable sets}

\begin{definition}
  A set $X$ is \emph{finite} if there is $n \in \NN$ and a bijection $f : n \to
  X$.

  A set $X$ is \emph{countable} if there is a surjection $f : \NN \to X$.
\end{definition}

\begin{proposition}
  Non-empty finite sets are countable.
\end{proposition}
\begin{proof}
  Suppose $x \in X$ and $f : n \to X$ was a bijection. Then
  \[
    g(m) :=
    \begin{cases}
      f(m) & ~m < n \\
      x & \text{ otherwise}
    \end{cases}
  \]
  is a surjection as desired.
\end{proof}

% \newpage

% \section{Exercises}

% \begin{exercise}[4.22]
%   Prove $\bigcap x$ is a set.
% \end{exercise}
% \begin{solution}
%   Observe that $\bigcap x = \{y \in \bigcup x \mid \forall z \in x.~y \in x\}$.
% \end{solution}

% \begin{exercise}[4.34]
%   Let $M$ be non-empty. Why is $\{A \times \{A\} \mid A \subseteq M, A \neq
%   \emptyset\}$ a set?
% \end{exercise}

% \begin{exercise}[4.36]
%   Why is the existence of the following sets paradoxical?
%   \begin{enumerate}[(a)]
%   \item $\{x \mid x \text{ a singleton}\}$
%   \item $\{x \mid x \text{ an ordered pair}\}$
%   \end{enumerate}
% \end{exercise}

% % Reference: https://randall-holmes.github.io/Drafts/parameterfree.pdf
% \begin{exercise}
%    In the lecture we presented ZF4, the axiom of separation, using parameters
%   \(p_1, \ldots, p_n\), i.e. \(\forall p_1, \ldots, p_n \forall x \exists y
%   \forall z.~z \in y \leftrightarrow (z \in x \wedge A(z, p_1, \ldots, p_n)) \)
%   where \(A\) is a formula whose only free variables are \(z\) and \(p_1, \ldots,
%   p_n\). However, the axiom can, equivalently, be stated without parameters, that
%   is ZF4*: \(\forall x \exists y \forall z.~z \in y \leftrightarrow (z \in x
%   \wedge A(z)) \) where \(A\) is a formula whose only free variable is \(z\).

% In this exercise you show that ZF4* entails ZF4 under ZF1-3 and ZF5-6:
% \begin{enumerate}
% \item
%   Use ZF1-3, ZF5-6 as well as ZF4* to prove the existence of the following set
%   for arbitrary $A$ and $B$:
%   \[
%     A \otimes B := \{[a, b] \mid a \in A, b \in B\}
%   \]
%   where $[a, b] = \langle \langle \{a\}, 1 \rangle, \langle \{b\}, 0
%     \rangle \rangle$.
%   Further, prove that it satisfies the ordered pair property: If $[a, b] = [a', b']$ then $a = a'$ and $b = b'$.
% \item
% Prove ZF4 using the above collection of axioms. Hint: To separate on a set \(X\) with parameters \(p_1, \ldots, p_n\), begin by applying ZF4* to \(X \otimes \{p_1\} \otimes \ldots \{p_n\}\).
% \end{enumerate}
% \end{exercise}
% \begin{solution}
%   \
%   \begin{enumerate}
%   \item Fix $A$ and define $A' := \{\{a\} \mid a \in A\} = \{X \in \pow(A) \mid
%     X \text{ a singleton}\}$ which exists by separation. Then let
%     \[
%       A_n := \{p \in \pow(\pow(A' \cup \{n\})) \mid \exists a.~p = \langle
%       \{a\}, n \rangle\}.
%     \]
%     Finally, fix
%     \[
%       A \otimes B := \{p \in \pow(\pow(A_1 \cup B_2)) \mid \exists a b.~p =
%       \langle \langle \{a\}, 1 \rangle, \langle \{b\}, 0 \rangle \rangle\}.
%     \]
%     and write $[a, b] := \langle \langle \{a\}, 1 \rangle, \langle \{b\}, 0
%     \rangle \rangle$.
%   \item Fix $A(-)$, the parameters $p_1, \ldots, p_n$ and the target of
%     separation $X$. First obtain
%     \[
%       Z = \{x \in X \otimes \{p_1\} \otimes \ldots \otimes \{p_n\} \mid \forall z
%       \forall q_1, \ldots, q_n.~x = [z, q_1, \ldots, q_n ] \wedge
%       A(z, q_1, \ldots, q_n)\}
%     \]
%     by parameter-free separation on \(X \times \{p_1\} \times \ldots \times
%     \{p_n\}\). Then obtain the desired $Y$ via
%     \[
%        Y = \{y \in X \mid \exists q_1, \ldots, q_n.~[ y, q_1, \ldots, q_n
%        ] \in Z\}.
%     \]
%   \end{enumerate}
% \end{solution}

% \begin{exercise}[3.8]
%   Let $n \neq \emptyset$, show that $0 \in n$.
% \end{exercise}

% \begin{exercise}[3.11]
%   Show that $m < n$ iff $m \subset n$.
% \end{exercise}

% \begin{exercise}[3.13]
%   Show if $n \in \NN$ and $x \in n$ then $x \in \NN$.
% \end{exercise}

\printbibliography{}

\end{document}
