\documentclass{whrartcl}

\title{Set theory: Week 3}
\author{Dominik Wehr, University of Gothenburg}
\date{Updated: \today}

\usepackage[inference]{semantic}
\usepackage{mathpartir}
\usepackage{xcolor}
\usepackage{url}
\usepackage{bussproofs}

% Fix \mu in the bibliography
\DeclareUnicodeCharacter{03BC}{$\mu$}
\usepackage[
  style=numeric-comp,
  sorting=nyt,
  maxbibnames=99,
  backref=false,
  doi=false,
  isbn=false,
  url=false,
  eprint=false,
  backend=biber
  ]{biblatex}
\bibliography{bibliography}

\newcommand{\NN}{\mathbb{N}}
\newcommand{\ZZ}{\mathbb{Z}}
\newcommand{\QQ}{\mathbb{Q}}
\newcommand{\RR}{\mathcal{R}}
\newcommand{\BB}{\mathbb{B}}
\newcommand{\PP}{\textbf{P}}
\newcommand{\DD}{\mathcal{D}}
\newcommand{\II}{\mathcal{I}}
\newcommand{\TT}{\mathcal{T}}
\newcommand{\Aa}{\mathcal{A}}
\newcommand{\Bb}{\mathcal{B}}
\newcommand{\Vv}{\mathcal{V}}
\newcommand{\WW}{\textbf{W}}
\newcommand{\LL}{\mathcal{L}}
\newcommand{\CC}{\mathcal{C}}
\newcommand{\pow}{\mathcal{P}}
\newcommand{\form}{\textsc{Form}}
\newcommand{\dom}{\text{dom}}
\newcommand{\ran}{\text{ran}}

\newcommand{\ol}[1]{\overline{#1}}
\newcommand{\abs}[1]{|#1|}
\newcommand{\uh}{\upharpoonright}
\newcommand{\irule}[3]{\inference[\textsc{#1}]{#2}{#3}}
\newcommand{\cls}[1]{\llbracket #1 \rrbracket}

\newcommand{\TODO}[1]{{\color{red}[TODO: #1]}}
\newcommand{\needcite}{{\color{red}[?]}}

\begin{document}

\maketitle

\section{Lecture 1}

\emph{Material:} WF of $\NN$, Set Recursion on $\NN$ (Sections 3.2, 3.3, 4.5)

\subsection{Well-orderedness}

\begin{definition}
  A set $X$ is \emph{well-ordered} by a linear order $\preceq \subseteq X \times
  X$ if for every $\emptyset \neq Y \subseteq X$ there is a least element of
  $Y$, i.e. $y \in Y$ such that $y \leq x$ for all $x \in Y$.
\end{definition}

\begin{proposition}
  $\NN$ is well-ordered by $\leq$.
\end{proposition}
\begin{proof}
  Fix $\emptyset \neq X \subseteq \NN$. And suppose $X$ had no least element. We
  show that $\{n \in N \mid \forall m \leq n.~m \not\in X\} = \NN$ by induction,
  i.e. that $X = \emptyset$ after all. For $0$ this boils down to it not being
  the least element of $Y$, which we have assumed. Thus, suppose all $m \leq n$
  were not in $B$ and consider $s(n)$. If $s(n) \in X$ then it would be its
  least element, hence $s(n) \not\in X$ and thus all $m \leq s(n)$ are not in
  $X$.
\end{proof}

\subsection{Recursion}

\begin{theorem}[Recursion]
  Fix $x_0 \in X$ and $h : X \to X$ then there exists a unique function $f :
  \NN \to X$ such that
  \[
    f(0) = x_0 \qquad \qquad f(s(n)) = h(f(n)) \text{ for every } n \in \NN
  \]
\end{theorem}
\begin{proof}
  \emph{Idea:} Show that we can approximate $f$ to arbitrary $n$ and then take
  their union.

  We call a function $g : s(n) \to X$ an $n$-germ for $n \in \NN$ if
  \begin{itemize}
  \item $g(0) = x_0$ and
  \item $g(s(m)) = h(g(m))$ for any $m < n$.
  \end{itemize}

  \begin{description}
  \item[Existence:] We prove that for every $n \in \NN$, there exists an
    $n$-germ by induction on $n$.
    \begin{description}
    \item[$0$:] Simply take $\{\langle 0, x_0 \rangle\}$.
    \item[$n \to s(n)$:] Suppose there was an $n$-germ $g : s(n) \to X$.
      Consider $g^* := g \cup \{\langle s(n), h(g(n)) \rangle\} : s(s(n)) \to X$. Then $g^*$ is
      an $s(n)$-germ because for any $m < s(n)$, either $m = n$ and $g^*(s(n)) =
      h(g(n)) = h(g^*(n))$ per construction, or $m < n$ and $g^*(s(m)) = g(s(m))
      = h(g(m)) = h(g^*(m))$.
    \end{description}
  \item[Uniqueness:] We prove that if $n, m \in \NN$ and $g : s(n) \to X$ and $g'
    : s(m) \to X$ are an $n$- and $m$-germ, respectively, then, setting $k :=
    n \cap m$, $g \uh s(k) = g' \uh s(k)$. Induction on $k$.
    \begin{description}
    \item[$0$:] Then $g(0) = x_0 = g'(0)$, meaning $g \uh 1 = g' \uh 1$.
    \item[$k \to s(k)$:] Then $g \uh s(s(k)) = g \uh s(k) \cup \{\langle s(k),
      g(s(k)) \rangle\}$. An analogous decomposition can be made for $g'$. Per
      IH we know that $g \uh s(k) = g' \uh s(k)$. It remains to show that
      $g(s(k)) = g'(s(k))$, but
      \[
        g(s(k)) = h(g(k)) = h(g'(k)) = g'(s(k)).
      \]
    \end{description}
  \end{description}

  We can now define
  \[
    f := \bigcup \{g \subseteq N \times X \mid \exists n \in \NN.~ g \text{ an }
    n\text{-germ}\} \subseteq N \times X.
  \]
  The inner set can be obtained via comprehension.
  We have to prove various properties of $f$.
  \begin{description}
  \item[$f$ a function:] For $n \in \NN$, let $\langle n, x \rangle, \langle n,
    x' \rangle \in f$. Then there are $m$-germ $g$ and $m'$-germ $g'$ such that
    $x = g(n)$ and $x' = g'(n)$. By uniqueness, $g(n) = g'(n)$.
  \item[Equations:] Follows from an induction using the $n$-germ properties.
  \item[Uniqueness:] Follows from equations and an induction.
  \end{description}
\end{proof}

\subsection{Arithmetic}

\begin{definition}
  For $m \in \NN$ we define $\text{add}_m : \NN \to \NN$ by recursion via
  \[
    \text{add}_m(0) := m \qquad \qquad \text{add}_m(s(n)) := s(\text{add}_m(n))
  \]
  and combine these into addition $+ : \NN \times \NN \to \NN$ by setting $n + m
  := \text{add}_m(n)$.

  For $m \in \NN$ we define $\text{mult}_m : \NN \to \NN$ by recursion via
  \[
    \text{mult}_m(0) := 0 \qquad \qquad \text{mult}_m(s(n)) := m + \text{mult}_m(n)
  \]
  and combine these into multiplication $\cdot : \NN \times \NN \to \NN$ by setting $n \cdot m
  := \text{mult}_m(n)$.
\end{definition}

\begin{lemma}
  If $n < m$ then $a + n < a + m$ for $a \in \NN$.
\end{lemma}
\begin{proof}
  Induction on $a$.
  \begin{description}
  \item[$0$:] We know that $0 + n = n < m = 0 + m$.
  \item[$a + 1$:] We know that $a + n < a + m$. Hence $s(a) + n = s(a + n) < s(a +
    m) = s(a) + m$.
  \end{description}
\end{proof}

\newpage

\section{Lecture 2}

\emph{Material:} Replacement and Foundation (Section 4.5)

\subsection{Replacement}

We would like to be able to define the following function: \emph{See assignment!!}
\[
  \Vv_0 := \NN \qquad \qquad \Vv_{s(n)} := \pow(\Vv_n)
\]
However, we cannot use recursion as above because we cannot prove that the
domain of $\Vv$ is a set.

\begin{definition}
  We fix the following additional axiom:
  \begin{description}
  \item[ZF7:] Replacement

    $(\forall x \exists y.~A(x,y) \wedge \forall y'.~A(x, y') \to y = y') \to
    \forall u \exists v \forall y.~y \in v \leftrightarrow (\exists x \in
    u.~A(x, y))$ where $A(-)$ has at most two free variables.

    If $A(-)$ is a class-level function then its image on any set $u$ is a set.
  \end{description}
\end{definition}

\begin{remark}
  ZF7 can also be stated using parameters, but as in the exercise class, these
  can also be handled using the Cartesian product.
\end{remark}

\begin{proposition}
  Let $X$ be a set and $A(-)$ be a class-level function on $X$. Then the
  function described by $A$ exists as $a$ with $\dom(a) = X$.
\end{proposition}
\begin{proof}
  First we obtain the domain $Y$ via replacement. Then take
  \[
    a := \{\langle x, y \rangle \in X \times Y \mid A(x, y)\}.
  \]
\end{proof}

\begin{example}
  We can now define a function $A^\bullet$ with:
  \[ A^0 := \{\emptyset\} \qquad \qquad A^{s(n)} := A \times A^n\]
  which computes the sets of $n$-tuples of $A$.
\end{example}

\begin{theorem}[Class recursion]
  Let $A(-)$ be a formula with two free variables such that
  \[
    \forall x \exists y.~A(x, y) \wedge \forall y'.~A(x, y') \to y = y'.
  \]
  For any set $x_0$ there exists a unique function $f$ with $\dom(f) = \NN$ such that
  \[
    f(0) = x_0 \qquad\qquad A(f(n), f(s(n))) \text{ for every } n \in \NN
  \]
\end{theorem}
\begin{proof}
  Define an $n$-germ as a function $g$ with domain $s(n)$ and
  \begin{itemize}
  \item $g(0) = x_0$ and
  \item $A(g(m), g(s(m)))$ for every $m < n$.
  \end{itemize}
  Prove existence, uniqueness and extension as before. Then define
  \[
    f := \bigcup \{g \mid n \in \NN, g \text{ an } n\text{-germ}\}.
  \]
  where the inner set exists because of replacement. Then close the proof as
  before.
\end{proof}

\subsection{Foundation}

\begin{definition}
  We fix the following additional axiom:
  \begin{description}
  \item[ZF9:] Foundation; \textcolor{red}{Warning! Incorrect in Goldrei!}

    $\forall x \exists y.~x \neq \emptyset \to (y \in x \wedge x \cap y = \emptyset)$

    Every non-empty set is well-founded, i.e.\ contains a $\in$-minimal element.
  \end{description}
\end{definition}

\begin{proposition}
  Let $z$ be a set. Then $z \not\in z$.
\end{proposition}
\begin{proof}
  Consider foundation with $x = \{x\}$, then $x \cap \{x\} = \emptyset$.
\end{proof}

\begin{theorem}[No infinite chains]
  Consider a function $f$ with $\dom(f) = \NN$. Then $f$ cannot be an infinite
  $\in$-chain, i.e.
  \[\ldots \in f(n) \in \ldots \in f(2) \in f(1) \in f(0).\]
\end{theorem}
\begin{proof}
  Simply apply foundation to $\ran(f)$: There must be $x \in \ran(f)$, and thus an
  $n \in \NN$ such that $x = f(n)$, such that $f(n) \cap \ran(f) = \emptyset$.
  But $f(s(n)) \in f(n) \cap \ran(f)$, a contradiction!
\end{proof}

\begin{remark}
  With this, we are done! The axioms ZF1-ZF9 are all of Zermelo-Fraenkel set theory.
\end{remark}


% \newpage

% \section{Exercises}

% \begin{exercise}[3.44]
%   Show that $(m + n) \setminus n \simeq m$.
% \end{exercise}
% \begin{proof}
%   Proof per induction on $m$.
%   \begin{description}
%   \item[$m = 0$:] Then $m + n = n$ and $n \setminus n = 0$.
%   \item[$m \to m + 1$:] Then $s(m) + n = s(m + n)$. We know that $n \subseteq m
%     + n$. Hence $s(m + n) \setminus n = (m + n) \cup \{m + n\} \setminus n = \{m
%     + n\} \cup (m + n \setminus n)$. Let $b : m \to m + n \setminus n$ be a
%     bijection, then we extend it into a bijection $b' : s(m) \to s(m + n)
%     \setminus n$ via
%     \[
%       b'(k) =
%       \begin{cases}
%         m + n & k = m \\
%         b(k) & k < m
%       \end{cases}.
%     \]
%   \end{description}
% \end{proof}

% \begin{exercise}[4.48]
%   Show that $\langle x, y \rangle = \{x, \{x, y\}\}$ works. \emph{Hint:} Use foundation.
% \end{exercise}
% \begin{proof}
%   Let $\langle x, y \rangle = \langle a, b \rangle$.
%   \begin{description}
%   \item[$x = a$:] We have $x \in \{a, \{a, b\}\}$. Suppose $x = \{a, b\}$ then
%     $a \in x$. Conversely, we have $a \in \{x, \{x, y\}\}$.
%     \begin{description}
%     \item[$a = x$:] Then $x = a \in x$.
%     \item[$a = \{x, y\}$:] Then $x \in a \in x$.
%     \end{description}
%   \item[$y = b$:] We have $\{a, b\} \in \{x, \{x, y\}\}$.
%     \begin{description}
%     \item[$\{a, b\} = x$:] This entails $x = a \in x$.
%     \item[$\{a, b\} = \{x, y\}$:] Suppose $b = x$ then $b = x = a$ and thus $\{b\} = \{x, y\}$, entailing $y = b$.
%     \end{description}
%   \end{description}
% \end{proof}

% \begin{exercise}[4.52]
%   Prove that $\hat{\NN} = \{\emptyset, \{\emptyset\}, \{\{\emptyset\}\},
%   \ldots\}$ exists.
% \end{exercise}
% \begin{proof}
%   Apply class-level recursion to obtain the function:
%   \[
%     f(0) = \emptyset \qquad f(n + 1) = \{f(n)\}.
%   \]
%   Then $\hat{\NN} = \dom(f)$.
% \end{proof}

\printbibliography{}

\end{document}
