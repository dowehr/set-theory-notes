\documentclass{whrartcl}

\title{Set theory: Week 7}
\author{Dominik Wehr, University of Gothenburg}
\date{Updated: \today}

\usepackage[inference]{semantic}
\usepackage{mathpartir}
\usepackage{xcolor}
\usepackage{url}
\usepackage{bussproofs}

% Fix \mu in the bibliography
\DeclareUnicodeCharacter{03BC}{$\mu$}
\usepackage[
  style=numeric-comp,
  sorting=nyt,
  maxbibnames=99,
  backref=false,
  doi=false,
  isbn=false,
  url=false,
  eprint=false,
  backend=biber
  ]{biblatex}
\bibliography{bibliography}

\newcommand{\NN}{\mathbb{N}}
\newcommand{\ZZ}{\mathbb{Z}}
\newcommand{\QQ}{\mathbb{Q}}
\newcommand{\RR}{\mathbb{R}}
\newcommand{\OO}{\mathbb{O}}
\newcommand{\BB}{\mathbb{B}}
\newcommand{\FF}{\mathcal{F}}
\newcommand{\PP}{\textbf{P}}
\newcommand{\DD}{\mathcal{D}}
\newcommand{\II}{\mathcal{I}}
\newcommand{\TT}{\mathcal{T}}
\newcommand{\Aa}{\mathcal{A}}
\newcommand{\Bb}{\mathcal{B}}
\newcommand{\Vv}{\mathcal{V}}
\newcommand{\WW}{\textbf{W}}
\newcommand{\LL}{\mathcal{L}}
\newcommand{\CC}{\mathcal{C}}
\newcommand{\pow}{\mathcal{P}}
\newcommand{\form}{\textsc{Form}}
\newcommand{\dom}{\text{dom}}
\newcommand{\ran}{\text{ran}}
\newcommand{\IS}{\text{IS}}
\newcommand{\ord}{\text{Ord}}
\newcommand{\Exp}{\text{Exp}}

\newcommand{\ol}[1]{\overline{#1}}
\newcommand{\abs}[1]{|#1|}
\newcommand{\uh}{\upharpoonright}
\newcommand{\irule}[3]{\inference[\textsc{#1}]{#2}{#3}}
\newcommand{\cls}[1]{\llbracket #1 \rrbracket}

\newcommand{\TODO}[1]{{\color{red}[TODO: #1]}}
\newcommand{\needcite}{{\color{red}[?]}}

\begin{document}

\maketitle

\section{Lecture 1}

\emph{Warning:} This week will go very fast. We have covered only 160 pages
of the book so far and this week will cover nearly 70 pages. We will
have to skip some proofs.

\subsection{Well-orders}

\begin{definition}
  Let $<\, \subseteq X \times X$ be a linear order:
  \begin{description}
  \item[irreflexive:] We never have $x < x$ for $x \in X$
  \item[transitive:] If $x < y$ and $y < z$ then $x < z$
  \item[linear:] Either $x < y$ or $x = y$ or $x > y$
  \end{description}

  \begin{enumerate}
  \item Let $x \in X$ then the \emph{initial segment of $x$} is $\IS_<(x) := \{y
    \in X \mid y < x \}$.
  \item Then $(X, <)$ is a well-order if it exhibits the \emph{well-ordering
      property}: For every $\emptyset \neq A \subseteq X$ there is an element $a
    \in A$ with $a \leq A$.
  \item Then $(X, <)$ exhibits $<$-induction if for every set $P \subseteq X$ such that
    \[
      \text{for every } x \in X \text{ if } \IS_<(x) \subseteq P \text{ then } x
      \in P
    \]
    we have that $P = X$.
  \end{enumerate}
\end{definition}

\begin{theorem}
  Let $<\, \subseteq X \times X$ be linear. Then it is a well-order iff it
  exhibits $<$-induction.
\end{theorem}
\begin{proof}
  \
  \begin{description}
  \item[$\to$:] Let $<$ be a well-order and let $P \subseteq X$ be such that for
    every $x \in X$ if $\IS(x) \subseteq P$ then $x \in P$. Suppose $P \neq X$,
    i.e. suppose there was $X \setminus P \neq \emptyset$. By the well-ordering
    property of $<$ there is a least $x \in X \setminus P$, meaning $\IS_<(x)
    \subseteq P$. But then $x \in P$ as well, leading to a contradiction.
  \item[$\leftarrow$:] We prove the following per $<$-induction:
    \[
      P := \{x \in X \mid \forall \emptyset \neq A \subseteq X.~x \in A \to \min(A) \text{
        exists}\} = X
    \]
    First, suppose $P = X$ and consider $\emptyset \neq A \subseteq X$. Because
    $\emptyset \neq A$ there is some $x \in A$. Because $x \in P$ we thus obtain
    $\min(A)$ as desired.

    To prove $P = X$ by $<$-induction, let $x \in X$ be such that $\IS_<(x)
    \subseteq P$ and consider $x \in A \subseteq X$. There are two cases to
    distinguish:
    \begin{description}
    \item[$\exists a \in A.~a < x$:] Then, because $a \in P$, we obtain that
      $\min(A)$ exists.
    \item[$x = \min(A)$:] Then we are done.
    \end{description}
  \end{description}
\end{proof}

\begin{theorem}
  Let $(X, <)$ be a well-order. Then exhibits $<$-recursion: For every three place formula $A$ such that
    \[
      \forall x \in X, y.~\exists z.~A(x, y, z) \wedge \forall z'.~A(x, y, z')
      \to z = z'
    \]
    there exists a function $f$ with $\dom(f) = X$ and such that for all $x \in
    X$ we have $A(x, f \uh x, f(x))$ where $f \uh x \coloneq \{(y, z) \in f \mid
    y < x\}$.
\end{theorem}
\begin{proofsketch}
  We proceed as usual for recursion theorems. For $x \in X$ an $x$-germ is a function
  $g$ with $\dom(g) = \IS_<(x) \cup \{x\}$ such that for all $y \in
  \dom(g)$ we have $A(y, g \uh y, g(x))$. We prove, per $<$-induction, the following
  two properties:
  \begin{itemize}
  \item For every $x \in X$ there exists an $x$-germ
  \item If $x, y \in X$ and $g, g'$ are an $x$- and $y$-germ, respectively, then
    $g \uh \min(x, y) = g' \uh \min(x, y)$
  \end{itemize}
  Then we obtain the following set via replacement
  \[
    G := \{g \mid g \text{ is an } x\text{-germ for } x \in X\}
  \]
  and observe that $f := \bigcup G$ is as desired.
\end{proofsketch}

\subsection{Introducing Ordinals}

\begin{remark}
  Recall: $n^+ := n \cup \{n\}$. What about $\NN^+ = \NN \cup \{\NN\}$?
  It is still a well-order wrt $\in$! Define
  \[
    \omega + 0 := \NN \qquad \omega + (n + 1) := (\omega + n)^+
  \]
  Then $\omega + \omega = \NN \cup \{\omega + n \mid n \in \NN\}$ is still a
  well-order! We shall study well-orders that are like it.
\end{remark}

\begin{definition}
  And ordinal $\alpha$ is a set such that:
  \begin{itemize}
  \item $(\alpha, \in)$ is a well-order and
  \item $\alpha$ is \emph{transitive:} If $\beta \in \alpha$ then $\beta
    \subseteq \alpha$.
  \end{itemize}
\end{definition}

\begin{lemma}
  Let $\alpha$ be an ordinal and $\beta \in \alpha$. Then
  \begin{enumerate}[(i)]
  \item $\beta = \IS_\in(\beta)$ and
  \item $\beta$ is an ordinal.
  \end{enumerate}
\end{lemma}
\begin{proof}
  \
  \begin{enumerate}[(i)]
  \item Observe that $\IS_\in(\beta) = \{\gamma \in \alpha \mid \gamma \in
    \beta\} \stackrel{\beta \subseteq \alpha}{=} \beta$.
  \item We know that $\beta$ is a $\in$-well-order because $\beta \subseteq \alpha$.
    For transitivity, let $\gamma \in \beta$ and observe that by transitivity of
    $\in$ we have $\delta \in \beta$ for any $\delta \in \gamma \in \beta$.
  \end{enumerate}
\end{proof}

\begin{lemma}
  Let $\alpha$ be an ordinal then so is $\alpha^+ := \alpha \cup \{\alpha\}$.
\end{lemma}
\begin{proof}
  We prove that $\alpha^+$ is transitive. Let $\beta \in \alpha^+$ then either
  $\beta \in \alpha$ and thus $\beta \subseteq \alpha \subseteq \alpha^+$ or
  $\beta = \alpha \subseteq \alpha^+$.
\end{proof}

\begin{lemma}
  Let $A$ be a set of ordinals. Then $\bigcup A$ is an ordinal.
\end{lemma}
\begin{proof}
  \
  \begin{description}
  \item[Transitive:] Let $\beta \in \bigcup A$ then $\beta \in \alpha \in A$ and
    thus $\beta \subseteq \alpha \subseteq A$.
\item[$\in$-well-order:] As usual, the union of linear orders is a linear order.
  For the well-ordering property, let $B \subseteq \bigcup A$ such that $\beta \in B$.
  Consider $B' := \{\alpha \in B \mid \alpha \in \beta\} \cup \{\beta\}$;
  because $\beta \in \gamma \in A$ there is $\min(B')$. By linearity, suppose
  there was $\delta \in B$ with $\delta \in \min(B')$ then $\delta \in B'$,
  contradicting the minimality of $\min(B')$.
  \end{description}
\end{proof}

\begin{corollary}
  Let $A$ be a set of ordinals. Then $A$ is an ordinal iff $A$ is transitive.
\end{corollary}
\begin{proof}
  Observe that if $A$ is a set of ordinals, then $A \subseteq \bigcup \{\alpha^+ \mid
  \alpha \in A\}$ because for any $\alpha \in A$ we have $\alpha \in \alpha^+
  \in \{\alpha^+ \mid \alpha \in A\}$.
\end{proof}

\begin{corollary}
  Let $\alpha, \beta$ be ordinals. Then $\alpha \in \beta$ or $\alpha = \beta$
  or $\beta \in \alpha$.
\end{corollary}
\begin{proof}
  Observe that $\alpha, \beta \in \alpha^+ \cup \beta^+$, the latter being an
  ordinal. The claim then follows from the linearity of $\alpha^+ \cup \beta^+$.
\end{proof}

\begin{corollary}
  There is no set $\OO$ of all ordinals.
\end{corollary}
\begin{proof}
  $\OO$ would be an ordinal, because it is transitive: Let $\alpha \in \OO$ then
  every $\beta \in \alpha$ is an ordinal and hence $\alpha \subseteq \OO$. But
  then $\OO \in \OO$, contradicting foundation.
\end{proof}

\subsection{Successors and limits}

\begin{definition}
  Let $\alpha$ be an ordinal. It is
  \begin{itemize}
  \item a \emph{successor} if there exists an ordinal $\beta \in \alpha$ such that $\alpha
    = \beta^+$.
  \item a \emph{limit} if for every ordinal $\beta \in \alpha$ we have $\beta^+
    \in \alpha$.
  \end{itemize}
\end{definition}

\begin{example}
  \
  \begin{enumerate}
  \item $0 \neq n \in \omega$ and $\omega + n$ for $n > 0$ are successors.
  \item $\omega$ and $\omega + \omega$ are limits.
  \end{enumerate}
\end{example}

\begin{theorem}
  Let $\alpha$ be an ordinal. Then it is a successor or a limit.
\end{theorem}
\begin{proof}
  We perform a case-distinction:
  \begin{description}
  \item[$\exists \beta \in \alpha.~\alpha = \beta^+$:] Then $\alpha$ is a successor.
  \item[$\forall \beta \in \alpha.~\alpha \neq \beta^+$:] Let $\beta \in \alpha$
    and suppose that $\alpha \in \beta^+$ (by linearity). That means that
    $\alpha \in \beta$ or $\alpha = \beta$. In either case, this violates foundation.
  \end{description}
\end{proof}

\begin{remark}
  It is somewhat controversial whether $0$ should be a limit or not. The book
  says it is not.
\end{remark}


\newpage

\section{Lecture 2}

\subsection{Ordinals are Ordinal Numbers}

\begin{definition}
  Let $(X, <_X)$ and $(Y, <_Y)$ be linear orders on $X, Y$, respectively. We call $f : X
  \to Y$ \emph{order-preserving} iff
  \[
    x <_X x' \text{ iff } f(x) <_Y f(x') ~~\text{for all } x, x' \in X.
  \]
  If $f$ is injective, it is an \emph{order-embedding}. If $f$ is a bijection,
  it as \emph{order-isomorphism}.
  When $(X, <_X)$ and $(Y, <_Y)$ are order isomorphic, we write $(X, <_X) \approx (Y,<_Y)$.
\end{definition}

\begin{theorem}
  For a WO $(X, <)$ there is an ordinal $\ord(X, <)$ such that
  $(X, <) \approx (\ord(X), \in)$.
\end{theorem}
\begin{proof}
  We define a recursive function $f$ with $\dom(f) = X$ as follows:
  \[
    A(x, f \uh x, z) := (f \uh x \text{ is a function } \to z = \ran(f \uh x))
    \wedge (f \uh x \text{ is no function } \to z = \emptyset)
  \]
  and fix $\ord(X, <) := \alpha := \ran(f)$.

  This is an order-isomorphism between $(X, <)$ and $(\alpha, \in)$.
  \begin{description}
  \item[surjection:] As $\alpha = \ran(f)$.
  \item[order $\to$:] Let $x < x'$ then $x \in \IS_<(x')$ and thus $f(x) \in
    \ran(f \uh x) = f(x')$.
  \item[injection:] Let $f(x) = f(x')$. By linearity, let w.l.o.g.\ $x < x'$
    then by the previous clause $f(x) \in f(x') = f(x)$, contradicting foundation.
  \item[order $\leftarrow$:] Let $f(x) \in f(x') = \ran(f \uh x)$, then there is
    $x'' \in \dom(f \uh x) = \IS_<(x')$ with $f(x) = f(x'')$. By injectivity we
    obtain $x = x'' < x'$.
  \end{description}

  It remains to prove that $\alpha$ is an ordinal.
  \begin{description}
  \item[$\forall x \in X.~f(x)$ is ordinal:] Per induction on $x$. Suppose $f(x) = \ran(f \uh x)$ was a
    set of ordinals. If it is transitive it is an ordinal. Let $\beta
    \in \ran(f \uh x)$. Then $\beta = \ran(f \uh x')$ for $x' < x$ and thus
    $\beta \subseteq \ran(f \uh x)$.
  \item[$\alpha$ transitive:] Let $\beta \in \alpha = \ran(f)$, meaning $\beta = \ran(f
    \uh x)$ for some $x \in X$, meaning $\beta = \ran(f \uh x) \subseteq \ran(f)$.
  \end{description}
\end{proof}

\begin{proposition}
  Let $\alpha$ and $\beta$ be ordinals such that $(\alpha, \in) \approx (\beta,
  \in)$. Then $\alpha = \beta$.
\end{proposition}
\begin{proof}
  Let $b : \alpha \to \beta$ be the order-isomorphism. We prove $b(\gamma) =
  \gamma$ for all $\gamma \in \alpha$ and thus that $\alpha \subseteq \beta$. An
  analogous argument yields $\beta \subseteq \alpha$.
  Thus let $\gamma \in \alpha$ such that $f(\delta) = \delta$ for all $\delta
  \in \gamma$. Then observe that
  \[
    f(\gamma) = \{\delta' \mid \delta' \in f(\gamma)\} \stackrel{\star}{=}
    \{f(\delta) \mid \delta \in \gamma\} \stackrel{IH}{=} \{\delta \mid \delta \in \gamma\} = \gamma
  \]
  where the $\star$-step follows from the fact that $(\alpha, \in) \approx
  (\beta, \in)$ (Exercise!).
\end{proof}

\begin{corollary}
  For well-orders $(X, <), (Y, <)$ holds: $(X, <) \approx (Y, <)$ iff $\ord(X,
  <) = \ord(Y, <)$.
\end{corollary}
\begin{proof}
  \
  \begin{description}
  \item[$\to$:] Then $(\ord(X, <), \in) \approx (X, <) \approx (Y, <) \approx
    (\ord(Y, <), \in)$ and thus $\ord(X, <) = \ord(Y, <)$.
  \item[$\leftarrow$:] Then $(X, <) \approx (\ord(X, <), \in) = (\ord(Y, <),
    \in) \approx (Y, <)$.
  \end{description}
\end{proof}

\subsection{Ordinal and order arithmetic}

\begin{definition}
  Let $\Omega$ be some ordinal. We can define arithmetical operations with
  domain $\Omega \times \Omega$ per recursion:
  \begin{align*}
    \alpha + 0 &= \alpha & \alpha \cdot 0 &= 0 & \alpha^0 &= 1 \\
    \alpha + \beta^+ &= (\alpha + \beta)^+ & \alpha \cdot \beta^+ &= (\alpha \cdot \beta) + \alpha & \alpha^{\beta^+} & = \alpha^\beta \cdot \alpha \\
    \alpha + \lambda &= \bigcup \{\alpha + \beta \mid \beta \in \lambda\} & \alpha \cdot \lambda &= \bigcup \{\alpha \cdot \beta \mid \beta \in \lambda\} & \alpha^\lambda & = \bigcup \{\alpha^\beta \mid \beta \in \lambda\}
  \end{align*}
  where $\lambda$ is a limit ordinal.
\end{definition}

\begin{remark}
  we would like to define operations on $\OO \times \OO$ where $\OO$ is the set
  of ordinals. However, we cannot do this as $\OO$ is not a set. Denote by
  $+_\Omega$ the $+$-operation above with domain $\Omega \times
  \Omega$. We can prove that $+_\Omega \subseteq +_{\Omega'}$ for
  $\Omega \in \Omega'$, i.e. the above induces a cumulative hierarchy of
  arithmetical operations. Informally, we thus write $\alpha + \beta$ to mean
  $\alpha +_\Omega \beta$ with $\Omega := \alpha^+ \cup \beta^+$, which `in the
  limit' agrees with all other $+_{\Omega'}$. The same is true for
  multiplication and exponentiation.
\end{remark}

\begin{remark}
  The operations are not commutative!
  \begin{align*}
    1 + \omega &= \bigcup \{1 + n \mid n \in \omega\} = \omega & \omega + 1 &= \omega^+\\
    2 \cdot \omega &= \bigcup \{2 \cdot n \mid n \in \omega\} = \omega & \omega \cdot 2 &= 0 + \omega + \omega = \omega + \omega
  \end{align*}
\end{remark}

\begin{definition}
  Let $(X, <_X), (Y, <_Y)$ be well-orders then their order sum $(X + Y, <_{X \oplus Y})$ is defined by
  \[
    X + Y := (X \times \{0\}) \cup (Y \times \{1\})
  \]
  and
  \[
    <_{X \oplus Y} := \{((x, 0), (x', 0)) \mid x <_X x'\} \cup \{((x, 0), (y,
    1)) \mid x \in X, y \in Y\} \cup \{((y, 1), (y', 1)) \mid y <_Y y'\}.
  \]
\end{definition}

\begin{proposition}[Goldrei; Theorem 8.8]
  If $(X, <_X), (Y, <_Y)$ are well-orders with $\alpha = \ord(X, <_X), \beta =
  \ord(Y, <_Y)$ then $(X + Y, <_{X \oplus Y})$ is a well-order with $\ord(X + Y,
  <_{X \oplus Y}) = \alpha + \beta$.
\end{proposition}

\begin{definition}
  Let $(X, <_X), (Y, <_Y)$ be well-orders then their order product $(X \times Y, <_{X \otimes Y})$ is defined by
  \[
    <_{X \otimes Y} := \{((x, y), (x', y')) \mid x, x' \in X, y <_Y y'\} \cup \{((x, y), (x',
    y)) \mid x <_X x', y \in Y\}
  \]
\end{definition}

\begin{proposition}[Goldrei; Theorem 8.10]
  If $(X, <_X), (Y, <_Y)$ are well-orders with $\alpha = \ord(X, <_X), \beta =
  \ord(Y, <_Y)$ then $(X \times Y, <_{X \otimes Y})$ is a well-order with $\ord(X \times Y,
  <_{X \otimes Y}) = \alpha \cdot \beta$.
\end{proposition}

\begin{definition}
  Let $(X, <_X), (Y, <_Y)$ be well-orders then their order exponential $(\Exp(X,
  Y), <_{X^Y})$ is defined by
  \[
    \Exp(X, Y) := \{f : Y \to X \mid \{y \in Y \mid f(y) \neq 0_X\} \text{ is finite} \}
  \]
  where $0_X = \min X$ and
  \[
    f <_{X^Y} g \text{ iff } \exists y \in Y.~f(y) < g(y) \wedge \forall y <
    y'.~f(y') = g(y').
  \]
\end{definition}

\begin{proposition}
  If $(X, <_X), (Y, <_Y)$ are well-orders with $\alpha = \ord(X, <_X), \beta =
  \ord(Y, <_Y)$ then $(\Exp(X, Y), <_{X^Y})$ is a well-order with $\ord(\Exp(X, Y),
  <_{X^Y}) = \alpha^\beta$.
\end{proposition}

\begin{theorem}[Cantor normal form]
  For any ordinal $\alpha$ there is a unique $n \in \NN$ and a unique sequence
  $\beta_1 \geq \ldots \geq \beta_n$ of ordinals such that
  \[
    \alpha = \omega^{\beta_1} + \ldots + \omega^{\beta_n}.
  \]
\end{theorem}

\printbibliography{}

\end{document}
