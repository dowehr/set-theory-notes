\documentclass{whrartcl}

\title{Set theory: Week 1}
\author{Dominik Wehr, University of Gothenburg}
\date{Updated: \today}

\usepackage[inference]{semantic}
\usepackage{mathpartir}
\usepackage{xcolor}
\usepackage{url}
\usepackage{bussproofs}

% Fix \mu in the bibliography
\DeclareUnicodeCharacter{03BC}{$\mu$}
\usepackage[
  style=numeric-comp,
  sorting=nyt,
  maxbibnames=99,
  backref=false,
  doi=false,
  isbn=false,
  url=false,
  eprint=false,
  backend=biber
  ]{biblatex}
\bibliography{bibliography}

\newcommand{\NN}{\mathbb{N}}
\newcommand{\RR}{\mathcal{R}}
\newcommand{\BB}{\mathbb{B}}
\newcommand{\PP}{\textbf{P}}
\newcommand{\DD}{\mathcal{D}}
\newcommand{\II}{\mathcal{I}}
\newcommand{\TT}{\mathcal{T}}
\newcommand{\Aa}{\mathcal{A}}
\newcommand{\Bb}{\mathcal{B}}
\newcommand{\WW}{\textbf{W}}
\newcommand{\LL}{\mathcal{L}}
\newcommand{\CC}{\mathcal{C}}
\newcommand{\pow}{\mathcal{P}}
\newcommand{\form}{\textsc{Form}}
\newcommand{\dom}{\text{dom}}
\newcommand{\ran}{\text{ran}}

\newcommand{\ol}[1]{\overline{#1}}
\newcommand{\abs}[1]{|#1|}
\newcommand{\uh}{\upharpoonright}
\newcommand{\irule}[3]{\inference[\textsc{#1}]{#2}{#3}}

\newcommand{\TODO}[1]{{\color{red}[TODO: #1]}}
\newcommand{\needcite}{{\color{red}[?]}}

\begin{document}

\maketitle

\section{Lecture 1}

\emph{Material:} Language of ST, Na\"ive ST

\subsection{Organizational matters}

\paragraph{Most matters}

My email: dominik.wehr@gu.se.

Lectures on Tuesday \& Wednesday at 10, exercise class on Thursday at 10.

Literature: \emph{Classic Set Theory} by Derek Goldrei. (Available online)

\paragraph{Examination}

There will be one sit down written examination. This is an individual closed
book exam and you are not allowed to bring any text, book, computer or other
device to the exam.

There will be 3 assignment sheets are not obligatory to pass the course, but we
strongly recommend students to take the opportunity to get feedback on written
solutions.

\subsection{Formal Language}
\emph{Reading:} Subsection 4.2

\begin{tabular}{c|c}
  Formula & Meaning \\
  \hline
  $x = y$ & ``$x$ equals $y$'' \\
  $x \in y$ & ``$x$ is element of $y$'' \\
  $\neg A$ & ``not $A$'' \\
  $A \wedge B$ & ``$A$ and $B$'' \\
  $A \vee B$ & ``$A$ or $B$'' \\
  $\forall x. A(x)$ & ``for every set $x$, $A(x)$ holds'' \\
  $\exists x. A(x)$ & ``there exists $x$ st. $A(x)$ holds''
\end{tabular}

\begin{example}
  \
  \begin{itemize}
  \item $\exists x. \forall y.~y \in x$: There is a universal set
  \item $\forall y. y \in x \to y = a \vee y = b$: $x = \{a, b\}$
  \item $\exists x. \forall y. \forall z. y \in x \wedge z \in x \to y = z$: At
    most one element in $x$
  \item $\forall a.~a \in x \leftrightarrow a \in y \vee a \in z$: $x = y \cup z$
  \item $\forall a.~a \in x \leftrightarrow a \in y \wedge a \in z$: $x = y \cap z$
  \item $\forall a.~a \in x \leftrightarrow a \in y \wedge a \not\in z$: $x = y \setminus z$
  \end{itemize}
\end{example}

\subsection{Na\"ive Set Theory}
\emph{Reading:} Subsection 4.1

\begin{definition}[Na\"ive Set Theory]
  Naive set theory is classical FOL + the following axiom scheme for every
  formula $A$ with free variable $y$ and parameters:
  \[
    \text{(Comprehension)}~~\exists x. \forall y.~y \in x \leftrightarrow A(y)
  \]
  We write $\{y : A(y)\}$ to denote said set.
\end{definition}

\begin{example}
  $\text{One} := \{x : x \text{ has at most one element}\}$ then $\{\text{One}\}
  \in \text{One}$
\end{example}

\begin{example}
  $\RR \coloneq \{x : x \not\in x\}$. Then:
  \begin{itemize}
  \item $\RR \not\in \RR$: Suppose $\RR \in \RR$ then $\RR \not\in \RR$ per
    def.\ of $\RR$. Contradiction!
  \item $\RR \in \RR$: Because $\RR \not\in \RR$.
  \end{itemize}
  \emph{Conclusion:} Na\"ive Set Theory is inconsistent (= we can derive a contradiction).
\end{example}

\paragraph{Plan:} Build a consistent set theory. That means:
\begin{itemize}
\item \emph{consistent:} The theory should not contain contradictions. Since
  Gödel's work, we know that we cannot (convincingly) prove this. However
  \begin{enumerate}[(a)]
  \item we believe that ZF(C) is consistent and
  \item nobody has yet managed to show its inconsistency.
  \end{enumerate}
\item \emph{set theory:} The theory will only speak about sets. Non-set objects,
  such as numbers, functions, etc.\ will be ``simulated'' by suitable sets.
\end{itemize}


\newpage
\section{Lecture 2}

\emph{Material:} ZF1-3, Ordered pairs, Union axiom (Sections 4.1-4.4)


\subsection{Zermelo-Fraenkel: Axioms 1--3}
\emph{Reading:} Subsection 4.3

\begin{definition}
  We fix the following axioms:
  \begin{description}
  \item[ZF1:] Extensionality

    $\forall x \forall y.~x = y \leftrightarrow \forall z.~z \in x
    \leftrightarrow z \in y$

    Two sets are precisely equal if they have the same elements.

  \item[ZF2:] Empty set

    $\exists x \forall y.~ y \not\in x$

    The empty set exists.

  \item[ZF3:] Pairing

    $\forall x \forall y \exists p \forall z.~z \in p \leftrightarrow (z = x
    \vee z = y)$

    For any $x,y$, the set $\{x, y\}$ exists.
  \end{description}
\end{definition}

\begin{proposition}[Uniqueness]
  The set $\emptyset$ and the set $\{x, y\}$ are always unique.
\end{proposition}
\begin{proof}
  \
  \begin{description}
  \item[$\emptyset$:] Let $x$ and $y$ have no elements. We prove $x = y$ using
    extensionality, i.e. $z \in x$ iff $z \in y$ for any $z$. Let $z \in x$. But
    as $x$ is empty, we also know that $z \not\in x$, a contradiction! The
    opposite direction follows analogously.
  \item[$\{x, y\}$:] Let $p, q$ be pairings of $x, y$. Observe that for any $z$
    \[
      z \in p \leftrightarrow z = x \vee z = y \leftrightarrow z \in q
    \]
    and hence $p = q$ by extensionality.
  \end{description}
\end{proof}

\begin{remark}
  \
  \begin{itemize}
  \item This justifies our use of the definite article
  \item This is a pattern that is generally true
  \end{itemize}
\end{remark}

\begin{proposition}[Singleton]
  For any $x$, the set $\{x\}$ exists.
\end{proposition}
\begin{proof}
  Simply pair $x$ with $x$. I.e.\ we obtain a set $p$ s.t.\ $z \in p
  \leftrightarrow z = x \vee z = x \leftrightarrow z = x$.
\end{proof}


\begin{remark}
  Any `collection of sets' is a \emph{class}, i.e.\ any one-place predicate. Not
  every class is a set, i.e. `$\{x \mid x \not\in x\}$' is a class but not a set.
\end{remark}

\subsection{Ordered pairs}
\emph{Reading:} Subsection 4.3

\emph{Problem:} We want a structure $\langle x, y \rangle$ such that $\langle x,
y\rangle = \langle a, b \rangle$ iff $x = a$ and $y = b$.

\emph{Non-solution:} $\langle x, y \rangle = \{x, y\}$. Problem: $\langle x, y
\rangle = \langle y, x \rangle$ then.

\begin{definition}[Ordered Pairs]
  We fix $\langle x, y \rangle := \{\{x\}, \{x, y\}\}$.
\end{definition}

\begin{proposition}
  If $x, y$ are sets then so is $\langle x, y \rangle$.
\end{proposition}
\begin{proof}
  Use Pairing.
\end{proof}

\begin{theorem}[Ordered Pair Property]
  $\langle x, y \rangle = \langle a, b \rangle$ iff $x = a$ and $y = b$
\end{theorem}
\begin{proof}
  The backwards direction is just equality.

  Suppose $\{\{x\}, \{x, y\}\} = \{\{a\}, \{a, b\}\}$:
  \begin{description}
  \item[$x = a$:] By extensionality, we know that $\{x\} = \{a\}$ or $\{x\} =
    \{a, b\}$.
    \begin{description}
    \item[$\{x\} = \{a\}$:] Then $x = a$ follows by extensionality.
    \item[$\{x\} = \{a, b\}$:] By extensionality, $a = x$.
    \end{description}
  \item[$y = b$:] Again, we know that $\{x, y\} = \{a\}$ or $\{x, y\} = \{a, b\}$.
    \begin{description}
    \item[$\{x, y\} = \{a, b\}$:] Then $y = a$ or $y = a$. If $y = b$ we are
      done. If $y = a$ with $x = a$ from above, we may continue as in the next
      case.
    \item[$\{x, y\} = \{a\}$:] Then $x = a = y$. But as $\{a, b\} = \{x, y\}$ or
      $\{a, b\} = \{x\}$, we have $\{a, b\} = \{a\}$, and thus $a = b$, in
      either case. Hence $y = a = b$.
    \end{description}
  \end{description}
\end{proof}

\begin{definition}
  We define $\langle x_1, \ldots, x_n \rangle := \langle x_1,
  \langle \ldots \langle x_n, \emptyset \rangle \rangle \rangle$.
\end{definition}

\subsection{Union axiom}
\emph{Reading:} Section 4.4

\begin{observation}
  We currently cannot prove that there are sets with more than 2 elements.
\end{observation}
\begin{corollary}
  We cannot prove that $A \times B$ for sets $A, B$ is a set.
\end{corollary}
\begin{proof}
  Let $A = \{\emptyset, \{\emptyset\}\}$ then $\abs{A \times A} = 4$.
\end{proof}

\begin{definition}
  We fix the following axiom:
  \begin{description}
  \item[ZF6:] Union

    $\forall x \exists y \forall z.~z \in y \leftrightarrow \exists w.~z \in
    w \wedge w \in x$

    The set $\bigcup x$ exists
  \end{description}
\end{definition}

\begin{proposition}[Union]
  If $x, y$ are sets then so is $x \cup y$.
\end{proposition}
\begin{proof}
   Observe that $x \cup y = \bigcup \{x, y\}$.
\end{proof}

\begin{proposition}[List sets]
  For $\{x_1, \ldots, x_n\}$ is a set for sets $x_1, \ldots, x_n$.
\end{proposition}
\begin{proof}
  Induction on $n$.
  \begin{description}
  \item[$n = 0$:] The empty set exists.
  \item[$n = m + 1:$] Then $I = \{x_1, \ldots, x_m\}$ is a set per IH. Then $\{x_1,
    \ldots, x_m, x_{m + 1}\} = I \cup \{x_{m + 1}\}$ (this requires another induction).
  \end{description}
\end{proof}

\begin{remark}
  For $A = \{\emptyset, \{\emptyset\}\}$ we can now define $A \times A$:
  \[
    A \times A := \{\langle \emptyset, \emptyset \rangle, \langle \emptyset,
    \{\emptyset\} \rangle, \langle \{\emptyset\}, \emptyset \rangle, \langle
    \{\emptyset\}, \{\emptyset\} \rangle\}
  \]
\end{remark}

% \newpage
% \section{Exercises}

% 4.3: 4.12, 4.14

% \begin{exercise}[4.12]
%   Let $x, y, z$ be sets.
%   \begin{enumerate}[(a)]
%   \item Show that $\langle x, y, z \rangle$ is unique.
%   \item Show that for $u, v, w$ with $\langle x, y, z \rangle = \langle u, v, w
%     \rangle$, $u = x$, $v = y$ and $w = z$.
%   \item Alternatively, we can take $\langle x, y, z \rangle := \langle \langle
%     x, y \rangle, z \rangle$. Does this work? Are ordered triples represented
%     the same way?
%   \end{enumerate}
% \end{exercise}

% \begin{exercise}[4.14]
%   Which of the following definitions of $\langle x, y, z \rangle$ work?
%   \begin{enumerate}[(a)]
%   \item $\{\{x\}, \{x, y\}, \{x, y, z\}\}$
%   \item $\langle \{x, y\}, {y, z} \rangle$
%   \item $\{\langle x, y \rangle, \langle y, z \rangle, \langle z, x \rangle\}$
%   \item $\{\langle x, y \rangle, \langle x, z \rangle, \langle y, z \rangle\}$
%   \item $\{\langle x, y \rangle, \langle y, z \rangle\}$
%   \item $\{\langle x, \langle x, y \rangle, \langle \langle y, z \rangle, z \rangle \rangle\}$
%   \end{enumerate}
% \end{exercise}

\printbibliography{}
\end{document}
